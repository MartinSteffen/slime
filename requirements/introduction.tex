\section{Introduction}
\label{sec:introduction}

The document describes informally the functionality of \Slime, a graphical
tool for editing and analyzing SFCs (\textit{\textbf{S}equentia\textbf{l}
  funct\textbf{i}on charts \textbf{m}odeling \textbf{e}nvironment}).


One crucial part of the implementation, around which most of the rest has
been arranged, is the \emph{abstract syntax} (cf.\ 
Section~\ref{sec:abstractsyntax}).


The rest of the documents sketches the parts of the project, each
implemented by one \emph{package} of the project. Especially, we describe
in first approximation
\begin{itemize}
\item the functionality offered by each package, and
\item the functionality expected from other packages.
\end{itemize}


As we intend to start \emph{early} with the \emph{integration}, the
required methods should be provided rather quickly without being (fully)
implemented (i.e., as \textit{stubs}). See also the time-line of the
project.

We provide as starting point a first implementation of the abstract syntax
(cf.\ Section~\ref{sec:abstractsyntax}) and a small textual printer in the
utilities package.



If from the perspective of a package, changes or extensions seem necessary
or desirable as far as the abstract syntax is concerned, the wish should be
uttered and justified as early as possible to all participants (and then
potentially implemented by us or the requester, if everyone agrees).

%%% Local Variables: 
%%% mode: latex
%%% TeX-master: "main"
%%% End: 
