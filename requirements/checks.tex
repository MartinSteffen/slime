\section{Checks}

\label{sec:checks}

\team{Thomas Richter, Karsten Stahl,
  \url{http://www.informatik.uni-kiel.de/\home{ms}}{Martin Steffen}, all others
  too}

Only syntactically correct systems can be meaningfully processed, in our
case \emph{simulated}. The task of this package is to check syntactical
consistency. The task comprises the \emph{definition} of what syntactical
correctness means, i.e., what is guaranteed/checked by this group upon
which the others can rely on.





\subsubsection*{Interface}

With the gui. The gui has to take care that the packages for
graph-placement, simulation, model-checking, code-generation \ldots are
handed over only checked syntax. What needs not to be checked are
``graphical lapses'', e.g., whether the nodes are placed one over the other
or similar things. 


\subsubsection*{Contract}



The checker assures well-typedness of expressions. The languages is rather
simple and the abstract syntax of Section~\ref{sec:abstractsyntax} only
features integers and booleans and no scopes. We assume that the language
is well-typed.  The types of the operators is shown in
Table~\ref{tab:types}.




\begin{table}[htbp]
  \centering
  \begin{displaymath}
    \begin{array}{rl}
      \text{operator/constant} & \text{type(s)}
      \\\hline
      \mathit{true},\mathit{false}
      & \Bool
      \\
      0,1,\ldots & \Int
      \\
      +,*,/ & \Int \times \Int \to \Int
      \\
      -     & \Int \times \Int \to \Int, \Int\to\Int
      \\
      <,>,\leq,\geq & \Int \times\Int \to \Bool
      \\
      =, \not= & \Int\times\Int \to \Bool, \Bool\times\Bool\to\Bool
      \\
      \neg & \Bool\to \Bool
    \end{array}
  \end{displaymath}
  \caption{Types}
  \label{tab:types}
\end{table}


Besides (and prior to) type-checking, the checker detects also various
situations, when the programs is considered to be ill-formed. The checker
guarantees, that
\begin{itemize}
\item the sfc is non-null, 
\item the (unique) initial step is contained in the set of steps,
\item all steps and transitions are non-null, 
\item no step occurs twice (by name), and
\item no transition has a non-existing step as source or target (by name)
\end{itemize}
Note that test of non-nullness is not done for parts other than mentioned
(for instance sap's etc.)  





%%% Local Variables: 
%%% mode: latex
%%% TeX-master: "main"
%%% End: 
