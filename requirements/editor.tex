\section{Editor}
\label{sec:editor}

\team{Yasin Taskin and Benjamin Bahnsen}


The graphical \emph{editor} for SFC's should support the following features:

\begin{itemize}
\item\textbf{construction:} The editor must be able to build-up a SFC
  |perhaps using some templates. A proposal how an SFC could look like is
  shown in Figure~\ref{fig:SFC}.
\item \textbf{save and load:} It must be able to store and reload programs
\item \textbf{select:} Parts of an SFC must be selectable; this is needed
  for other actions
\item \textbf{delete and copy:} selected parts can be deleted and copied.
\item \textbf{highlight}: parts of the displayed SFC, especially steps and
  transitions, can be marked as highlighted.  
\end{itemize}



\subsubsection*{Interfaces}

With the Gui (Section~\ref{sec:gui}), where the division of work between
gui and editor should be discussed. Furthermore with the simulator
(Section~\ref{sec:simulator}), concerning the highlighting.  \iffalse Ob es
auch ein De-Highlighten gibt, ist noch ungekl�rt. 

In any case: a method such as \texttt{highlight\_state}, given either
\begin{itemize}
\item the name of the state, oder
\item the state as object.
\end{itemize}
The choice must be agreed upon with the simulator or the gui werden,
depending on who calls the method.\fi

\iffalse
Further methods offered by the editor tZugriff auf
\begin{enumerate}
\item das gespeicherte SFC (f�r den Simulator aus
  Abschnitt~\ref{sec:simulator}) und
\item Zugriff auf den Status (unver�andert, gespeichert \ldots)
\end{enumerate}
\fi

An important interface (as for all other packages) is with the abstract
syntax package. To support the graphical representation, the abstract
syntax classes are equipped with \emph{coordinates,} the meaning and the
representation is to be discussed among editor and the
graphical-placement-package. 




%%% Local Variables: 
%%% mode: latex
%%% TeX-master: "main"
%%% End: 
