
\section{Graphical layout}
\label{sec:layout}


\team{Andreas Niemann}

The editor allows to draw SFC's free-handedly. Besides that it should be
possible to calculate coordinates of the transition system automatically.
To this end, a \emph{graphical layout algorithm} must be implemented, that
takes care of displaying the SFC in a readable manner.


\subsubsection*{Interface}
Gui and editor. The layouter may assume checked syntax. What the meaning of
the coordinates is concerned, this must be agreed upon with the editor
%  Da der Editor genau ein
% Fenster pro Prozess bereitstellt, m�ssen nur diese vom Platzierungsgruppe
% positioniert werde, nicht ganze Programme. Gui �bernimmt die
% Benutzerf�hrung in eigener Regie (nur ein Prozess soll positioniert werden,
% z.B., derjenige, dessen Fenster den Fokus hat), oder alle Prozesse sollen
% positioniert werden.

\iffalse
Angebote: eine Methode \texttt{position\_sfc}, die ein SFC in abstrakter
Syntax nimmt und ihn mit Koordinaten zur�ckgibt. Ob dies ebenfalls ein
Objekt der abstrakten Syntax ist oder einer anderen Datenstruktur, wurde
noch nicht festgelegt (siehe die Diskussion in Zusammenhang mit dem Editor
in Abschnitt~\ref{sec:editor}).

F�r den Anfang sei davon ausgegangen, da� alle Steps \emph{gleich gro�}
seien und Kanten als Geraden dargestellt werden.
\fi



\iffalse
\paragraph{Erweiterungsm�glichkeiten:} In einem ersten Schritt sollen
\emph{die Steps} plaziert werden, und die Transitionen als
\emph{Geraden} dazwischen. Falls Zeit ist, kann man versuchen,
\emph{gebogene} Transitionen zeichnen (d.h. auch berechnen!) zu lassen.
Sonstige Erweiterungsm�glichkeiten: Steps verschiedener Gr��en,
Ber�cksichtigung der Gr��e der Labels etc.
\fi











%%% Local Variables: 
%%% mode: latex
%%% TeX-master: "main"
%%% End: 
