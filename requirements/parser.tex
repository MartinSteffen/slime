
\section{Parser}
\label{sec:parser}


\team{Marco Wendel}




The tool should support a simple, non-graphical input language, to allow a
textual program specification. The textual specification is without
graphical information; this information can be calculated by the layout
package.

This module parses the textual input and generates an abstract syntax tree.
The implementation uses \textsl{JLex} und \textsl{CUP}.

\subsubsection*{Interface}
With the gui (Section~\ref{sec:gui}), providing a method 
\texttt{parse\_file}.

\iffalse
 Der Parameter ist ein String,
welcher die Datei bezeichnet, die das Programm enth�lt.  Die Dateien sollen
als Standard-Extension \texttt{.slime}-besitzen. Der Parser kann die
Ausnahme \texttt{Parser\_Exception} werfen. W�nschenswert ist, da"s der
Parser zumindest die Zeilennummer des Fehlers in der Ausnahme zur�ckgibt.
\fi

\iffalse
%A further interface is with the\emph{Editor} (Section~\ref{sec:editor})
gefordert: Das Parserpaket soll f�r den Editor das \emph{parsen} eines
\emph{Ausdruckes} (also einer \texttt{absynt.Expr}) bereitstellen. Die
Eingabe soll ein \emph{String} sein. Bei Fehlschlag soll eine Ausnahme
geworfen werden.
\fi





%%% Local Variables: 
%%% mode: latex
%%% TeX-master: "main"
%%% End: 
