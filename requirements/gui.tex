\section{Graphical user interface  (Gui)}
\label{sec:gui}

\team{Norbert Boeck}

\Slime{} is built from various components interacting with the user.
There's an integrating top-level packages responsible for the following tasks:

\begin{itemize}
\item\textbf{Start:} When invoking a \Slime-session, a ``window'' appears
  which allows to activate the various sub-components of the system.
\item\textbf{Dependency:} Not all interactions are sensible all situations,
  for instance: a simulation can be started only on a syntactically correct
  program. The task is, to define the dependencies between the packages and
  implement them in the tool.
\item\textbf{Session management:} (2nd priority) It should be possible to
  save a session (opened windows, loaded files, chosen options \ldots).  It should be possible to reload a saved session.
\end{itemize}
The user interface \emph{integrates} all other packages. Thus, the one
responsible for the GUI should be especially aware of inconsistencies
between the packages and react upon detected violations.

\subsubsection*{Interfaces}

With all other packages, cf.\ the corresponding description there.


\iffalse
\subsubsection*{Design-Entscheidungen}
\begin{itemize}
\item Die Oberflache wird mit \textsl{swing}-Komponenten aufgebaut; das
  gleiche gilt f�r den Editor.
\item Die Oberfl�che wird mit dem Werkzeug \textsl{Forte 2.0} aufgebaut,
  das gleiche gilt f�r den Editor.
\end{itemize}
\fi








%%% Local Variables: 
%%% mode: latex
%%% TeX-master: "main"
%%% End: 
