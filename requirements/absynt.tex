
\section{Abstract syntax}
\label{sec:abstractsyntax}

\team{Karsten Stahl,
  \url{http://www.informatik.uni-kiel.de/\home{ms}}{Martin Steffen}, and
  all others}

The following \emph{extended BNF}-notation specifies the \emph{abstract
  syntax} as common intermediate data represenation for the project. Modulo
some naming conventions (capitalization), the \Java-implementation is
straightforward. Each non-terminal is represented as a separate class.
Alternatives, specified by $\bnfbar$, are subclasses of the \emph{abstract
  class,} to which they build the alternative. The entries of the middle
collum constitute the \emph{fields} of the classes. The constructors of the
classes are conventionally fixed by the fields of the class (up to the
order of the arguments.\footnote{There are exceptions to this rule, notably
  for the (\Slime-)types in the expressions. The type-fields are not
  included in the constructors. The corresponding fields will be set
  later.}  The \emph{lists} of the EBNF are implemented as
\texttt{java.lang.LinkedList}.  Graphical position information, relevant
only for the editor and the layout group, is omitted in the EBNF.


\medskip

\inputcode{sfc-absynt.txt}



%%%%%%%%%%%%%%%%%%%%%%%%%%%%%%%%%%%%%%%%%%%%%%%%%%%%%%%%%%%%%%%%%%%%%
%% $Id: absynt.tex,v 1.3 2002-07-10 05:53:53 swprakt Exp $
%%%%%%%%%%%%%%%%%%%%%%%%%%%%%%%%%%%%%%%%%%%%%%%%%%%%%%%%%%%%%%%%%%%%%
%%% Local Variables: 
%%% mode: latex
%%% TeX-master: "main"
%%% End: 
