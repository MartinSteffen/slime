\section{Semantics}
\label{sec:semantics}

The section informally describes the semantics of \textit{Sequential Function
  Charts (SFC's)}, as realized in the tool \Slime.  The semantics is defined
for successfully \emph{checked} SFC's (cf.\ Section~\ref{sec:checks});
unchecked SFC's don't have a meaning.  Especially, the simulator, which
realizes the semantics, can assume checked syntax.



\subsection{Sequential Function Charts}

We explain the semantics with the help of the example from
Figure~\ref{fig:SFC}.


\ifweb
\begin{figure}[htbp]
  \centering
  \begin{rawhtml}
    <BR>
    <DIV ALIGN=center>
    <IMG SRC="sfc-figure.gif">
    </DIV><BR>  
  \end{rawhtml}  
  \caption{SFC}
  \label{fig:SFC}
\end{figure}

\else
\begin{figure}[htbp]
  \centering
  \input{sfc-figure}
  \caption{SFC}
  \label{fig:SFC}
\end{figure}

%\includegraphics[height=10cm,clip=]{sfc-figure1.ps}
\fi


The SFC's consist of nodes, called \emph{steps}, to which \emph{actions} are
associated, and \emph{transitions} between steps, decorated with boolean
\emph{guards.} Always, one ore more of the steps are active and the actions
associate with this active steps are executed within one cycle.  The
transition from $s_1$ to both $s_2$ and $s_3$ (with double horizontal line) is
a \emph{parallel} branching: if this transition is taken, $s_1$ is deactivated
and both $s_2$ and $s_3$ get activated.  Since this is one transition, and
each transition has exactly one guard, the guard is labeled on the upper part
of the transition.

In contrast, the ``branching'' from $s_3$ to $s_5$ and $s_6$ is no real
branching, it is just an abbreviation for two different transitions: one
leading from $s_3$ to $s_5$, the other leading from $s_3$ to $s_6$.
Therefore, the guards are labeled to the lower parts, since each transition
has exactly one guard.

The topmost step (marked specifically) is \emph{initial.}  The ''N'' on the
left-hand side of the actions is a \emph{qualifier}, stating that the action
is to be executed in each cycle in which the step is active. There are other
qualifiers, too, but we will neglect them unless we find good reasons for
taking them into account.

The behavior of an SFC during one \emph{cycle} is as follows.
\begin{enumerate}
\item Reading inputs from the environment.
\item Executing the actions from the active steps.  This is done in two steps
  as follows:
  \begin{enumerate}
  \item Assemble all active actions (as a set, so each action appears at most
    one time).
  \item Execute the assembled actions in an arbitrary order.
  \end{enumerate}
\item Evaluate the guards.\footnote{If one does not allow propositions like
    \emph{step\_1\_is\_active}, taking a transition can only disable another
    transition by deactivating its source steps, the guards will remain true.}
\item Take transition(s) (if possible).
\item Write outputs.
\end{enumerate}
The cycle is executed repeatedly.  The parts for \emph{reading inputs} and
\emph{writing outputs} are irrelevant for us, as we consider closed systems
only, i.e., systems whose variables are changed only by the system itself, but
not by the outside.

Each transition is equipped by a \emph{guard,} i.e., a boolean expression. A
transition can be taken only if the guard evaluates to true, and, if all the
source steps of the transition are active.  We do not enforce the target steps
to be disabled.\footnote{If during a run a transition is taking which enters
  an already active step, the SFC is either built using a strange programming
  style or there is a mistake.  But in principle, there seems to be no reason
  to enforce target steps to be deactive.}

If more than one step is active in a parallel branch, the execution of the
corresponding action is chosen \emph{non-deterministically}. This means,
they can be executed in an arbitrary order (\emph{interleaving semantics}).

There is a second source of non-determinism, namely the set of actions
associated to the active steps.  These actions will be first assembled, and
then non-deterministically executed.  Each action will only be executed once,
even if an action is associated to two different active steps.

Consequently, a program may have a number of different execution runs.  The
simulator could realize the different runs in that it asks the user, in which
order the actions should be performed, and which transition should be taken if
several are possible.  An alternative is, to determine the order by a random
generator.

The transition from $s_4$ and $s_7$ to $s_8$ closes the parallel branch again.
Such a transition can be taken only, if \emph{all} source steps are active. In
other words, this transition can be taken if it's guard evaluates to true and
furthermore both $s_4$ and $s_7$ are active.



\subsection{States}
\label{sec:states}

The global state of a program is given by the assignment to all variables and
the set of all active steps.



%%% Local Variables: 
%%% mode: latex
%%% TeX-master: "main"
%%% End: 
