\section{Semantics}
\label{sec:semantics}

The section informally describes the semantics of \textit{Sequential
  Function Charts (SFC's)}, as realized in the tool \Slime.  The semantics
is defined for successfully \emph{checked} SFC's (cf.\ 
Section~\ref{sec:checks}); unchecked SFC's don't have a meaning.
Especially, the simulator, which realizes the semantics, can assume checked
syntax.

\subsection{Sequential Function Charts}

We explain the semantics with the help of the example from
Figure~\ref{fig:SFC}.







\ifweb
\begin{figure}[htbp]
  \centering
  \begin{rawhtml}
    <BR>
    <DIV ALIGN=center>
    <IMG SRC="sfc-figure.gif">
    </DIV><BR>  
  \end{rawhtml}  
  \caption{SFC}
  \label{fig:SFC}
\end{figure}

\else
\begin{figure}[htbp]
  \centering
  \input{sfc-figure}
  \caption{SFC}
  \label{fig:SFC}
\end{figure}

%\includegraphics[height=10cm,clip=]{sfc-figure1.ps}
\fi





The SFC's consist of nodes, called \emph{steps}, to which \emph{actions} are
associated, and \emph{transitionen} between steps, decorated with boolean
\emph{guards.} Always, one ore more of the steps are active and the actions
associate with this active steps are executed within one cylcle.  The
transition from $s_1$ to both $s_2$ and $s_3$ (with double horizontal line) is
a \emph{parallel} branching: if this transition is taken, 
$s_1$ is deactivated and both $s_2$ and $s_3$ get activated.

The topmost step (marked specifically) is \emph{initial.}  The ''N'' on the
left-hand side of the actions is a \emph{qualifier}, stating that the action
is to be executed in each cycle in which the step is active. There are other
qualifiers, too, but we neglect them for the teme being,
Qualifier, die wir aber erst einmal vernachl�ssigen.

The behavior of an SFC during one \emph{cycle} is as follows.
\begin{enumerate}
\item reading inputs from the environment
\item executing the actions from the active steps
\item evaluate the guards
\item take transition(s) (if possible)
\item write outputs
\end{enumerate}
The cycle is executed repeatedly.  The parts for \emph{reading inputs} and
\emph{writing outputs} are irrelevant for us, as we consider closed systems
ony, i.e., systems whose variables are changed only by the system itself, but
not by the outside.

Each transition is equipped by a \emph{guard,} i.e., a boolean expression. A transition can be taken only if the guard evaluates to true.

If more than one step is active in a parallel branch, the execution of the
corresponding action is chosen \emph{non-deterministically}. This means,
they can be executed in an arbitrary order (\emph{interleaving semantics}).
Consequently, a program may have a number of different execution runs.  The
simulator could realize the different runs in that it asks the user, in
which order the actions should be performed. An alternative is, to determie
the order by a random generator.


The transition from $s_4$ and $s_5$ to $s_8$ closes the parallel branch
again.  Such a transition can be taken only, if \emph{all} source, steps
are active. In other words, this transition can be taken if it's guard
evaluates to true and furthermore both $s_4$ und $s_5$ are active.


\subsection{States}
\label{sec:states}


The global state of a program is given by the assignment to all variables and
the set of all active steps.





%%% Local Variables: 
%%% mode: latex
%%% TeX-master: "main"
%%% End: 
