\section{Semantics}
\label{sec:semantics}

The section informally describes the semantics of \textit{Sequential Function
  Charts (SFC's)}, as realized in the tool \Slime.  The semantics is defined
for successfully \emph{checked} SFC's (cf.\ Section~\ref{sec:checks});
unchecked SFC's don't have a meaning.  Especially, the simultor and the
model-checker (Section~\ref{sec:simulator} and \ref{sec:assert}), which
realize the semantics, can assume checked syntax

\subsection{Sequential Function Charts}

We explain the semantics with the help of the example from
Figure~\ref{fig:SFC}.







\ifweb
\begin{figure}[htbp]
  \centering
  \begin{rawhtml}
    <BR>
    <DIV ALIGN=center>
    <IMG SRC="sfc-figure.gif">
    </DIV><BR>  
  \end{rawhtml}  
  \caption{SFC}
  \label{fig:SFC}
\end{figure}

\else
\begin{figure}[htbp]
  \centering
  \input{sfc-figure}
  \caption{SFC}
  \label{fig:SFC}
\end{figure}

%\includegraphics[height=10cm,clip=]{sfc-figure1.ps}
\fi





The SFC's consist of nodes, called \emph{steps}, to which \emph{actions} are
associated, and \emph{transitionen} between steps, decorated with boolean
\emph{guards.} Always, one ore more of the steps are active and the actions
associate with this active steps are executed within one cylcle.  The
transition from $s_1$ to both $s_2$ and $s_3$ (with double horizontal line) is
a \emph{parallel} branching: if this transition is taken, 
$s_1$ is deactivated and both $s_2$ and $s_3$ get activated.

The topmost step (marked specifically) is \emph{initial.}  The ''N'' on the
left-hand side of the actions is a \emph{qualifier}, stating that the action
is to be executed in each cycle in which the step is active. There are other
qualifiers, too, but we neglect them for the teme being,
Qualifier, die wir aber erst einmal vernachl�ssigen.

The behavior of an SFC during one \emph{cycle} is as follows.
\begin{enumerate}
\item reading inputs from the environment
\item executing the actions from the active steps
\item evaluate the guards
\item take transition(s) (if possible)
\item write outputs
\end{enumerate}
The cycle is executed repeatedly.  The parts for \emph{reading inputs} and
\emph{writing outputs} are irrelevant for us, as we consider closed systems
ony, i.e., systems whose variables are changed only by the system itself, but
not by the outside.

Die Transitionen sind mit einem \emph{Guard} ausgestattet sein, einem
\emph{booleschen Ausdruck.}  Eine Transition kann nur genommen werden,
falls sich der Guard zu \emph{true} evaluiert.

Sind aufgrund einer parallelen Verzweigung \emph{mehrere} Steps aktiv, so
erfolgt die Ausf�hrung der zugeh�rigen Aktionen nichtdeterministisch, d.h. sie
sind in beliebiger Reihenfolge m�glich (\emph{Interleaving-Semantik}).
Folglich gibt es unter Umst�nden eine Vielzahl verschiedener L�ufe eines
SFC's, abh�ngig von diesen Ausf�hrungsreihenfolgen.  Der Simulator soll dies
dadurch realisieren, dass er nach Wahl des Benutzers diesen fragt in welcher
Reihenfolge die Aktionen ausgef�hrt werden sollen, oder aber die Reihenfolge
per Zufallsgenerator festlegt.

Die Transition von $s_4$ und $s_5$ zu $s_8$ schlie�t die paralelle Verzweigung
wieder.  Solche Transitionenen k�nnen nur genommen werden, wenn \emph{alle}
Quell-Steps aktiv sind.  Folglich kann diese Transition nur genommen werden
kann, wenn ihr Guard zu \emph{true} evaluiert wird, und ferner die beiden
Steps $s_4$ und $s_5$ aktiv sein.


\subsection{States}
\label{sec:states}


The global state of a program is given by the assignment to all variables and
the set of all active steps.





%%% Local Variables: 
%%% mode: latex
%%% TeX-master: "main"
%%% End: 
