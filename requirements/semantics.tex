\section{Semantics}
\label{sec:semantics}

The section describes the semantics of \textit{Sequential Function Charts
  (SFC's)}, as (to be) realized in the tool \Slime.  The semantics is
defined for successfully \emph{checked} SFC's (cf.\ 
Section~\ref{sec:checks}); unchecked SFC's don't have a meaning.
Especially, the simulator, which realizes the semantics, can assume checked
syntax.

\subsection{Informal semantics}
\label{sec:semantics.informal}

We start informally and explain the semantics with the help of the example
from Figure~\ref{fig:SFC}.


\ifweb
\begin{figure}[htbp]
  \centering
  \begin{rawhtml}
    <BR>
    <DIV ALIGN=center>
    <IMG SRC="sfc-figure.gif">
    </DIV><BR>  
  \end{rawhtml}  
  \caption{SFC}
  \label{fig:SFC}
\end{figure}

\else
\begin{figure}[htbp]
  \centering
  

% Macros for SFC pictures

\newcommand{\GUARD}[1][]{%
 \put(-1.5,0){\linethickness{0.2\unitlength}\line(1,0){3}}%
              \put(2.5,0){\makebox(0,0)[l]{#1}}}
\newcommand{\STEP}[1][]{{%
% \put(-2.5,-2.5){\rule{5\unitlength}{5\unitlength}}%
 \put(-2.5,-2.5){\linethickness{0.2\unitlength}%
                 \framebox(5,5){#1}}}}
\newcommand{\STEPinitial}[1][]{{%
% \put(-2.5,-2.5){\rule{5\unitlength}{5\unitlength}}%
 \put(-2.5,-2.5){\linethickness{0.2\unitlength}%
                 \framebox(5,5){#1}}}
 \put(-2.2,-2.2){\linethickness{0.2\unitlength}%
                 \framebox(4.4,4.4){}}}
\newcommand{\LINE}{\linethickness{0.15\unitlength}\line}
\newcommand{\VECTOR}{\thicklines\linethickness{0.15\unitlength}\vector}

% Mark a step as active
\newcommand{\Mark}{\put(-3,-3){\linethickness{0.5\unitlength}%
                   \framebox(6,6){}}}

% Action block
\newcommand{\AB}[3][\put(2.5,1){\line(1,0){2}}]{\linethickness{0.15\unitlength}
  #1
  \put(4.5,-0.5){\rule{12\unitlength}{3\unitlength}}
  \put(4.5,2.5){\line(1,0){12}}\put(4.5,-0.5){\line(1,0){12}}
  \put(4.5,-0.5){\line(0,1){3}}\put(7.0,-0.5){\line(0,1){3}}
  \put(16.5,-0.5){\line(0,1){3}}
  \put(5.75,1){\makebox(0,0){\small\texttt{#2}}}
  \put(11.75,1){\makebox(0,0){\small\textit{#3}}}
}

% Mark an action as active
\newcommand{\MarkA}{\put(7,-0.5){\linethickness{0.4\unitlength}%
                    \framebox(9.5,3){}}}

% Action block with timing constraint
\newcommand{\ABt}[3][]{\linethickness{0.15\unitlength}
  \put(2.5,1){\line(1,0){2}}
  \put(4.5,-1){\rule{14.5\unitlength}{4\unitlength}}
  \put(4.5,3){\line(1,0){14.5}}\put(4.5,-1){\line(1,0){14.5}}
  \put(4.5,-1){\line(0,1){4}}\put(9.5,-1){\line(0,1){4}}
  \put(19,-1){\line(0,1){4}}
  \put(7,2.7){\makebox(0,0)[t]{\small\texttt{#2}}}
  \put(7,-0.8){\makebox(0,0)[b]{\tiny\textrm{#1}}}
  \put(14.25,1){\makebox(0,0){\small\textit{#3}}}
}

% Mark a timed action as active
\newcommand{\MarkAt}{\put(9.5,-1){\linethickness{0.4\unitlength}%
                     \framebox(9.5,4){}}}



\newcommand{\INSTEP}[1]{\parbox{3cm}{\texttt{~~~~%
        \begin{tabular}{|c|l|} \hline
          N & #1 \\ \hline
        \end{tabular}
}}}
\newcommand{\INGUARD}[1]{{\texttt{~~#1}}}


\unitlength1.8mm%
\begin{picture}(44.5,75)(-32,-15)
  \put(  0, 55){\put(  0,-2.5){\LINE(0,-1){6}}
    \put(  0,-5.5){\GUARD{\INGUARD{true}}}
    \put(-12,-8.5){\LINE(1,0){24}}
    \put(-12,-9.5){\LINE(1,0){24}}
    \put(-12,-8.5){\LINE(0,-1){4}}
    \put( 12,-8.5){\LINE(0,-1){4}}
    \put( 2,0){\LINE(1,0){2}}
    \STEPinitial[$s_1$]{\INSTEP{act1}}
    }
  \put(-12, 40){\put(  0,-2.5){\LINE(0,-1){25}}
    \put(  0,-14.5){\GUARD{\INGUARD{x and y}}}
    \put( 2,0){\LINE(1,0){2}}
    \STEP[$s_2$]{\INSTEP{act2}}
    }
  % oberster im rechten zweig
  \put( 12, 40){\put(  0,-2.5){\LINE(0,-1){4}}
    \put( -5,-6.5){\LINE(1,0){10}}
    \put( -5,-6.5){\LINE(0,-1){6}}
    \put( -5,-9.5){\GUARD{\INGUARD{y}}}
    \put(  5,-6.5){\LINE(0,-1){6}}
    \put(  5,-9.5){\GUARD{\INGUARD{not y}}}
    \put( 2,0){\LINE(1,0){2}}
    \STEP[$s_3$]{\INSTEP{act3}}
    }
  % splitten im rechten zweig
  \put( 7, 25){\put(  0,-2.5){\LINE(0,-1){6}}
    \put(  0,-5.5){\GUARD{\INGUARD{true}}}
    \STEP[$s_5$]{}
    }
  \put( 17, 25){\put(  0,-2.5){\LINE(0,-1){6}}
    \put(  0,-5.5){\GUARD{\INGUARD{true}}}
    \STEP[$s_6$]{}
    }
  % zusammenfuehrung rechts nach splitten
  \put( 7, 19){\put(  0,-2.5){\LINE(1,0){10}}
    \put(  5,-2.5){\LINE(0,-1){4}}
    }
  \put(  12, 10){
    \STEP[$s_7$]{}
    \put(  0,-2.5){\LINE(0,-1){4}}
    }

  % unterer step im linken zweig + parallel-konstrukt schliessen
  \put(-12, 10){\put(  0,-2.5){\LINE(0,-1){4}}
    \put(  0,-5.5){\LINE(1,0){24}}
    \put(  0,-6.5){\LINE(1,0){24}}
    \put( 12,-6.5){\LINE(0,-1){6}}
    \put( 12,-9.5){\GUARD{\INGUARD{true}}}
    \STEP[$s_4$]{}
    }
  % unterster step + nach oben
  \put(  0, -5){\put(  0,-2.5){\LINE(0,-1){6}}
    \put(  0,-5.5){\GUARD{\INGUARD{true}}}
    \put(  0,-8.5){\LINE(-1,0){20}}
    \put(-20,-8.5){\LINE(0,1){68.5}}
    \put(-20,60  ){\VECTOR(1,0){17.5}}
    \STEP[$s_8$]{}
    }
  \put(-50,48){%
    \parbox{4cm}{\noindent\textbf{Deklarationen} \\[1ex]
      \begin{tabular}{|l|l|l|} \hline
        \texttt{x} & \texttt{bool} & \texttt{false} \\
        \texttt{y} & \texttt{bool} & \texttt{false} \\
        \texttt{z} & \texttt{bool} & \texttt{false} \\ \hline
      \end{tabular}
      }
    }
  \put(-50,25){%
    \parbox{4cm}{\noindent\textbf{Aktionen} \\[1ex]
      \begin{tabular}{|l|l|} \hline
        \texttt{act1} & \texttt{x\,:=\,false} \\
        \texttt{act2} & \texttt{y\,:=\,x} \\
        \texttt{act3} & \texttt{x\,:=\,not x; y\,:=\,x} \\ \hline
      \end{tabular}
      }
    }
\end{picture}


%%% Local Variables: 
%%% mode: latex
%%% TeX-master: "main"
%%% End: 

  \caption{SFC}
  \label{fig:SFC}
\end{figure}

%\includegraphics[height=10cm,clip=]{sfc-figure1.ps}
\fi


The SFC's consist of nodes, called \emph{steps}, to which \emph{actions} are
associated, and \emph{transitions} between steps, decorated with boolean
\emph{guards.} Always, one ore more of the steps are active and the actions
associate with this active steps are executed within one cycle.  The
transition from $s_1$ to both $s_2$ and $s_3$ (with double horizontal line) is
a \emph{parallel} branching: if this transition is taken, $s_1$ is deactivated
and both $s_2$ and $s_3$ get activated.  Since this is one transition, and
each transition has exactly one guard, the guard is labeled on the upper part
of the transition.

In contrast, the ``branching'' from $s_3$ to $s_5$ and $s_6$ is no real
branching, it is just an abbreviation for two different transitions: one
leading from $s_3$ to $s_5$, the other leading from $s_3$ to $s_6$.
Therefore, the guards are labeled to the lower parts, since each transition
has exactly one guard.

The topmost step (marked specifically) is \emph{initial.}  The ''N'' on the
left-hand side of the actions is a \emph{qualifier}, stating that the action
is to be executed in each cycle in which the step is active. There are other
qualifiers, too, but we will neglect them unless we find good reasons for
taking them into account.

The behavior of an SFC during one \emph{cycle} is as follows.
\begin{enumerate}
\item Reading inputs from the environment.
\item Executing the actions from the active steps.  This is done in two steps
  as follows:
  \begin{enumerate}
  \item Assemble all active actions (as a set, so each action appears at most
    one time).
  \item Execute the assembled actions in an arbitrary order.
  \end{enumerate}
\item Evaluate the guards.\footnote{If one does not allow propositions like
    \emph{step\_1\_is\_active}, taking a transition can only disable another
    transition by deactivating its source steps, the guards will remain true.}
\item Take transition(s) (if possible).
\item Write outputs.
\end{enumerate}

Each transition is equipped by a \emph{guard,} i.e., a boolean expression.
A transition can be taken only if the guard evaluates to true, and, if all
the source steps of the transition are active.  We do not enforce the
target steps to be disabled.\footnote{If during a run a transition is
  taking which enters an already active step, the SFC is either built using
  a strange programming style or there is a mistake.  But in principle,
  there seems to be no reason to enforce target steps to be inactive.}

If more than one step is active in a parallel branch, the execution of the
corresponding action is chosen \emph{non-deterministically}. This means,
they can be executed in an arbitrary order (\emph{interleaving semantics}).

There is a second source of non-determinism, namely the set of actions
associated to the active steps.  These actions will be first assembled, and
then non-deterministically executed.  Each action will only be executed once,
even if an action is associated to two different active steps.

Consequently, a program may have a number of different execution runs.  The
simulator could realize the different runs in that it asks the user, in which
order the actions should be performed, and which transition should be taken if
several are possible.  An alternative is, to determine the order by a random
generator.

The transition from $s_4$ and $s_7$ to $s_8$ closes the parallel branch again.
Such a transition can be taken only, if \emph{all} source steps are active. In
other words, this transition can be taken if it's guard evaluates to true and
furthermore both $s_4$ and $s_7$ are active.



\ifweb\else
\subsection{Formal semantics}
\label{sec:semantics.formal}

In this section, we define the operational semantics for SFCs as realized
in \Slime. An SFC-program is as a tuple $(S,s_\init,T,\Var,\act)$, where
$S$ is the (finite) set of steps, $s_i \in S$ the initial step, and $T$ the
set of transitions, where $T \subseteq 2^S \times \Expr \times 2^S$ and
$\Expr$ is the set of expressions as given by the abstract syntax. For
transitions $S_s \edge{g} S_t$, we assume the following restrictions: the
sets of sources and targets non-empty and disjoint, and either the the set
sources contains just one step, or else the set of targets, i.e., $S_s
\not= \emptyset \not= S_t$, $S_s \intersect S_t = \emptyset$, and
$\sizeof{S_s} = 1$ or $\sizeof{S_s} = 1$.

We write $s, s' \ldots$ for typical elements steps and $t,t', t_1, \ldots$
for typical transitions, and we will also write $S_s \edge{g} S_t$ for
transitions, where $S_s$ and $S_t$ are the sets of source and target steps,
respectively, and $g$ is a boolean expression, the \emph{guard} of the
transition. The set $\Var$ is the set of \emph{variables} where we silently
assume all variables and expressions to be well typed.\footnote{This
  corresponds to the agreement, that the simulator can rely on
  \emph{checked} syntax.}  By the typing, some variables are marked as
input or output variables or both, and we write $\Varin$ and $\Varout$ for
the corresponding subsets of $\Var$.  Variables that are neither input nor
output variable we call \emph{local} variable from the set $\Varloc$.  The
last component $\act \of S \to 2^\Act$ finally attaches a set of actions to
each steps.\footnote{The example from above in Figure~\ref{fig:SFC} is
  simplified insofar, as it contains at most \emph{one} action per step.}
We will use the function analogously also on sets of steps.


\subsubsection{Stores and configurations}





The global \emph{store} contains the values of all variables and is
modeled as a mapping $\Store = \Var \to D$, where $D$ stands for the data
domain and again we will assume the values in the store to be type
consistent; we use $\eta$ as typical element of $\Store$. A
\emph{configuration} $\gamma \OF 2^S\times \Store \times \Status$ of a
program is characterized by a set of active steps together with a store.
The third component of a configuration of type $\Status$ is used to control
the various phases of the system behavior and can be either $\inputstatus$
(``input''), or $\calcstatus{A}$ (``calculating'') where $A$ is a set of
actions, or $\outputstatus{T}$ (``output''), where $T$ is a set of
transitions.




\subsubsection{Operational semantics}


The operational rules are shown in Figure~\ref{tab:operationalsemantics},
specifying the labeled transition relation $\trans{l}$ between system
configurations. The labeled transitions $\transin{\vec{v}}$ and
$\transout{\vec{v}}$ are used to mark reading the input and writing the
output variables; all other transitions are unlabeled and internal.
Initially, the system starts in its initial step waiting for input, i.e., 
\emph{initial} configuration $\gamma_\init$ is given by
$(\{s_\init\},\eta_\init,\inputstatus)$, where the initial store
$\eta_\init$ evaluates all booleans to $\lfalse$ and all integers to $0$.

An execution cycle starts by reading the input (cf.\ rule \rn{Input}), when
the system is currently in the waiting state. After updating the store
$\eta$ by assigning values to all input variable as read from the
environment, and goes into the next stage. In doing so, it collects all
active actions given by $\act(\sigma)$ into the status.  In the next stage,
the system executes non-deterministically one action after the other as
given in rule \rn{Act}.  In this rule $\sem{}{a} \OF \Store \to \Store$ is
the state-transforming effect of the statement belonging to the action $a$,
i.e., $\sem{}{a}(\eta)$ is the store after executing $a$ when starting in
$\eta$.  Since actions contain only sequences of variable assignments, we
omit a formal definition of the update semantics of actions. When no action
is left, all currently enabled transitions are collected, i.e., the
transitions whose source steps are all active and whose guards evaluate to
$\ltrue$ (cf.\ rule \rn{No-Act}). The rule \rn{Trans} calculates the set of
active successor steps by working off the enabled transitions from
$\outputstatus{T}$ one after the other, while disabled transitions are
discarded by rule \rn{No-Trans}.\footnote{The formulation in \rn{No-Act}
  collecting all enabled transitions and afterwards calculating the
  successors with \rn{Trans} and \rn{No-Trans} rests on two assumptions:
  First, the guards are side-effect free and especially there are no
  expressions checking for activeness of steps. Secondly, for the division
  of work between \rn{Trans} and \rn{No-Trans} to be sensible, it must be
  an invariant of the system that a transition discarded by \rn{No-Trans}
  would not become enabled by enlarging the set of active steps in
  \rn{Trans}. In case of a structured use of parallelism (fork/join), where
  especially sequential dependencies between the parallel parts are
  avoided, this assumption will hold. This could be a task of the
  check-group.} When the transitions are worked-off, output is written in
the last stage of the cycle and the system enters the phase where it waits
for new input again (cf.\ rule \rn{Output}).


\begin{table}[htbp]
  \begin{ruleset}
    
\infrule{Input}{
  \eta' = \eta\setto{\vec{x}}{\vec{v}}
  \andalso
  \vec{x} = \Varin
  }{
  (\sigma,\eta,\waitinput) \transin{v}
  (\sigma,\eta',\doneinput{\act(\sigma)})
  }

\\

\infrule{Act}{%
  a \in A
  \andalso
  \eta' = \sem{}{a}(\eta)
  }{
  (\sigma,\eta,\doneinput{A}) 
  \trans{}
  (\sigma,\eta',\doneinput{A\without a})
  }

\infrule{No-Act}{
  \andalso
  T = \{
  S_s \edge{g} S_t \in T
  \suchthat
  S_s \subseteq \sigma
  \land
  \sem{\eta}{g} = \ltrue \}
  }{
  (\sigma,\eta,\doneinput{\emptyset})
  \trans{}
  (\sigma,\eta,\finished{T})
  }

\\

\infrule{Next-Steps}{%
  t =  S_s \edge{g} S_t
 \in T
  \andalso
  S_s  \subseteq \sigma
  \andalso
  \sigma' = (\sigma \union S_t) \without S_s
  }{
  (\sigma,\eta,\finished{T})
  \trans{}
  (\sigma',\eta,\finished{T \without t})
  }

\\

\infrule{Output}{%
  \vec{v} = \sem{\eta}{\vec{x}}
  \andalso
  \vec{x} = \Varout
  }{%
  (\sigma,\eta, \finished{\emptyset}) \transout{\vec{v}}
  (\sigma,\eta,\waitinput)
  }

%%% Local Variables: 
%%% mode: latex
%%% TeX-master: "main"
%%% End: 
  
  \end{ruleset}
  \caption{Operational semantics}
  \label{tab:operationalsemantics}
\end{table}

\fi%% Formal semantics not in HTML-Form






%%% Local Variables: 
%%% mode: latex
%%% TeX-master: "main"
%%% End: 
