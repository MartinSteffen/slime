\section*{Snapshots}
\label{sec:snapshots}


The \emph{baselines} or \emph{snapshots} are archived here for quick
reference as \texttt{slime\_v[x].jar}, where \texttt{[x]} is the number or
the tag of the snapshot. Developers with access to the source repository
can retrieve those snapshots of course with cvs.\footnote{The command is
  \texttt{cvs update -r [tag]}. Be careful with this command, don't
  generate development branches without knowing.}


To \emph{execute} one of the archived snapshots, save it at some convenient
place, set the \texttt{CLASSPATH} (or use the \texttt{-classpath} option)
of \texttt{java}) to include the saved jar-file + the required versions of
\texttt{JLex} and \texttt{java\_cup}, and do\footnote{Since the archive is
  not stand-alone in that it requires additionally at run-time the lexer
  and the parser generator, the archive is not executable with the
  \texttt{-jar} option.}

\begin{center}
  \texttt{java <slimeversion>.jar slime.Main}
\end{center}


The \emph{required} version is \texttt{jdk1.4,} earlier version may cause
trouble, later ones are not tested.


\medskip


\begin{tabular}{llll}
  \\\hline
  &
  archive/date
  &
  explanation
  &
  command
  \\\hline 
  4. & \url{snapshots/slime.jar}{18.\ July 2002}
  &
  snapshot ``endofsemester'': last-minute integration
  &
  \texttt{java slime.jar}
  \\

  3. & \url{snapshots/slime\_v3.jar}{10.\ July 2002}
  &
  snapshot 3: compiles, but not integrated
  &
  \texttt{java -jar slime\_v3.jar}
  \\

  2. & \url{snapshots/slime\_v.secondcompilation-sep.jar}{9.\ July 2002}
  &
  second compilation (separately)
  &
  \texttt{java slime.editor.EditorInFrame}
  \\
  1. & \url{snapshots/slime\_v.firstcompilaton-sep.jar}{3.\ July 2002}
  &
  first compilation (separately), all packages on board
  &
  \texttt{java slime.editor.EditorInFrame}
  \\
\end{tabular}



%%%%%%%%%%%%%%%%%%%%%%%%%%%%%%%%%%%%%%%%%%%%%%%%%%%%%%%%%%%%%%%%%%%%%%%%
%% $Id: snapshots.tex,v 1.8 2002-07-18 13:14:10 swprakt Exp $
%%%%%%%%%%%%%%%%%%%%%%%%%%%%%%%%%%%%%%%%%%%%%%%%%%%%%%%%%%%%%%%%%%%%%%%%
%%% Local Variables: 
%%% mode: latex
%%% TeX-master: "main"
%%% End: 
