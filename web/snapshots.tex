\section*{Snapshots}
\label{sec:snapshots}


The \emph{baselines} or \emph{snapshots} are archived here for quick
reference as \texttt{slime\_v[x].jar}, where \texttt{[x]} is the number or
the tag of the snapshot. Developers with access to the source repository
can retrieve those snapshots of course with cvs.\footnote{The command is
  \texttt{cvs update -r [tag]}. Be careful with this command, don't
  generate development branches without knowing.}


To \emph{execute} one of the archived snapshots, save it at some convenient
place, set the classpath to point to the jar-archve, and invoke the
interpreter, i.e., Java's virtual machine as indicated in the table below.
The \emph{required} version is \texttt{jdk1.4,} earlier version may cause
trouble, later ones are not tested.

For instance, after downloading, you can do

\begin{verbatim}
  java -classpath ./slime_v[x].jar <command>} 

\end{verbatim}
to start the tool. Alternatively, you can set the
\texttt{CLASSPATH}-variable appropriatly, in which case the
\texttt{-classpath}-option is not needed.

\medskip


\begin{tabular}{llll}
  \\\hline
  &
  archive/date
  &
  explanation
  &
  command
  \\\hline 
  2. & \url{snapshots/slime\_v.secondcompilation-sep.jar}{9.\ July 2002}
  &
  second compilation (separately)
  &
  \texttt{java slime.editor.EditorInFrame}
  \\
  1. & \url{snapshots/slime\_v.firstcompilaton-sep.jar}{3.\ July 2002}
  &
  first compilation (separately), all packages on board
  &
  \texttt{java slime.editor.EditorInFrame}
  \\
\end{tabular}




%%% Local Variables: 
%%% mode: latex
%%% TeX-master: "main"
%%% End: 
