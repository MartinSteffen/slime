\section*{Baselines}
\label{sec:Baselines}


Die \emph{Baselines} oder \emph{Snapshots} sind hier zur schnelleren
Referenz als \emph{Javaarchiv} \texttt{slime\_v[x].jar} festgehalten, wobei
\texttt{[x]} die Nummber des Schnappschusses ist. Als Entwickler mit
Zugriff auf das Quellcoderepositorium kann man die Entwicklungsschritte sie
nat�rlich mittels \cvs{} wieder zur�ckholen. Zum Ausf�hren der Archivs
speichere man die Datei \texttt{slime\_v[x].jar} an einen geeigneten Platz,
und rufe

\begin{verbatim}
  java -classpath [geeigneter_platz]/slime_v[x].jar gui.Gui
\end{verbatim}
auf; alternativ kann man auch fest
\begin{verbatim}
 export CLASSPATH=[geeigneter_platz]/slime_v[x].jar
\end{verbatim}
setzen, dann kann man sich die option \texttt{-classpath} schenken.


\medskip

\iffalse
\begin{tabular}{llll}
  3. & \url{baselines/slime\_v3.0.jar}{Baseline 3}
  &
  Parser an Bord
  &
  1. Juli 2001
  \\
  2. & \url{baselines/snot\_v2.0.jar}{Baseline 2}
  &
  Simulator an Bord
  &
  13. Juni 2001
  \\
  1. & \url{baselines/snot\_v1.0.jar}{Baseline 1}
  &
  erste Gesamtkompilation
  &
  5. Juni 2001
\end{tabular}
\fi


%%% Local Variables: 
%%% mode: latex
%%% TeX-master: "main"
%%% End: 
