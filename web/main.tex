


\newif\ifweb\webtrue    %%%%%% htmlonly geht irgendwie nicht
\newif\ifsolution\solutionfalse
\newif\ifworking\workingfalse


%%% Local Variables: 
%%% mode: latex
%%% TeX-master: "main"
%%% End: 


\documentclass[11pt,german]{article}
\usepackage{hevea}
%\usepackage{url}
\usepackage{babel}
\usepackage{epsfig}
\usepackage{tfheader}
\usepackage{a4wide}
\usepackage[latin1]{inputenc}

%\newcommand{\kommentar}[1]{[{\small\em #1}]

\newcommand{\Slime}{\textsc{Slime}}
\newcommand{\cvs}{cvs}
\newcommand{\Java}{\textsc{Java}}
\newcommand{\SMV}{\textsc{Smv}}
\newcommand{\Cplusplus}{C$^{++}$}
\newcommand{\javadoc}{\textsc{javadoc}}

\newcommand{\Int}    {\mathit{Int}}
\newcommand{\Bool}   {\mathit{Bool}}


\newcommand{\team}[1]{\textbf{Responsible:} #1\bigskip{}}



\newcommand{\bnfdef}{::=}
\newcommand{\bnfbar}{\ensuremath{\mid}}
\newcommand{\of}    {\ensuremath{\mathrel{:}}}
\newcommand{\without}{\backslash}
\newcommand{\ltrue}   {\mathit{true}}
\newcommand{\lfalse}  {\mathit{false}}

\newcommand{\suchthat}{\mathrel{\mid}}
\newcommand{\union}  {\cup}
\newcommand{\intersect}  {\cap}
\newcommand{\sizeof}[1]  {{\mid}{#1}{\mid}}
\newcommand{\sem}    [2]        {[\![#2]\!]_{#1}}      %semantics

\newenvironment{diagram}{\begin{displaymath}}{\end{displaymath}}

%\newcommand{\inputcode}[2][Code]   {
%  {\small
%  \mbox{}
%  \newline
%  \mbox{}
%  \hrulefill
%  \verbatiminput{#1/#2.java}
%  \hrulefill}}




\newcommand{\inputcode}[2][.]{
  {\footnotesize
  \mbox{}
  \newline\nopagebreak{}
  \mbox{}
  \hrulefill
  \lstinputlisting{#1/#2}
  \hrulefill}}

\newenvironment{code}{%
%  \small\mbox{}\nopagebreak{}\mbox\hrulefill{}
  {\begin{lstlisting}{}}}{%
  {\end{lstlisting}}}





%%%%%%%%%%%%%%%%%%%%%%%%%%%%%%%%%%%%%%% macros for semantics of SFC's

%%% RULES 


\newenvironment{ruleset}{
  \mbox{}\noindent\hrulefill\begin{displaymath}
    \renewcommand{\arraystretch}{1.6}\begin{array}[b]{l}}{
  \end{array}\end{displaymath}\hrulefill\mbox{}}




\newcommand{\treefont}{\small}
\newcommand{\leaf}[1]{{\begin{array}{c} #1 \end{array}}}
\newenvironment{proofleaf}{\begin{array}{c}}{\end{array}}

\newcommand{\namedruletree}[3]{{
     \treefont
     \prooftree 
       {\leaf{#2}}
     \justifies
       #3
     \using \rn{#1}
     \endprooftree
     }}                             

\newcommand{\ruletree}[2]{\namedruletree{}{#1}{#2}}


\newcommand{\rn}[1]{\mbox{\textsc{#1}}}   %% determines the font
\newcommand{\andalso}{\quad\quad}
\newcommand{\infrule}[3]{\namedruletree{#1}{#2}{#3}\andalso}
\newcommand{\scinfrule}[4]{\namedruletree{\mbox{$#4$}\quad\quad #1}{#2}{#3}\andalso}
\newcommand{\infax}[2]{
  \treefont
  #2\quad\quad \rn{#1}\andalso
  \andalso
  }
\newcommand{\scinfax}[3]{
  \treefont
  #2\quad \mbox{$#3$\quad\quad#1}
  \andalso
}




%%% SFCs
\newcommand{\init}          {\mathit{init}}
\newcommand{\act}           {\mathit{act}}
\newcommand{\Act}           {\mathit{Act}}
\newcommand{\States}        {\mathit{States}}
\newcommand{\Status}        {\mathit{Status}}
\newcommand{\Steps}         {\mathit{Steps}}
\newcommand{\Store}         {\mathit{Store}}
\newcommand{\Var}           {\mathit{Var}}
\newcommand{\Varin}         {\Var_{\mathit{in}}}
\newcommand{\Varout}        {\Var_{\mathit{out}}}
\newcommand{\Varloc}        {\Var_{\mathit{loc}}}

\newcommand{\src}           {\mathit{src}}
\newcommand{\target}        {\mathit{tar}}
\newcommand{\guard}         {\mathit{guard}}
\newcommand{\Expr}          {\mathit{Expr}}
\newcommand{\OF}            {\mathrel{:}}

\newcommand{\setto}[2]             {{\scriptstyle [#1\mathop{\mapsto}#2]}}  %setto{x}{v} : [x|->v]  {}
\newcommand{\substfor}[2]   {}

\newcommand{\inputstatus}       {\mathsf{I}}%{\mathit{WAITINPUT}}
\newcommand{\calcstatus}[1]     {\mathsf{C}(#1)}%{\mathit{DONEINPUT(#1)}}
\newcommand{\outputstatus}[1]   {\mathsf{O}(#1)}%{\mathit{FINISHED(#1)}}









%\newenvironment{diagram}{\begin{displaymath}}{\end{displaymath}}


\newcommand{\edge}[1]            {\longrightarrow_{#1}} %

%%% semantical steps
\newcommand{\trans}[1]             {\rightarrow_{#1}}
\newcommand{\transin}[1]           {\trans{?#1}}
\newcommand{\transout}[1]          {\trans{!#1}}








%%% Local Variables: 
%%% mode: latex
%%% TeX-master: t
%%% End: 



\title{{\huge\bf \textsl{S}equentia\textsl{l} Funct\textsl{i}on Charts
    \textsl{M}odeling \textsl{E}nvironment (aka.\ \Slime)}}
%\author{}
%\url{}{Martin Steffen}}
\date{}



\ifweb

\htmlfoot{\hrulefill{}
  {\footnotesize Pages last (re-)generated \today}}
\renewcommand{\@bodyargs}{bgcolor="white" alink="red" vlink="\#407999"  link="\#7070ff"}  
\fi




\begin{document}
\vspace{-2cm}


%\pagestyle{empty}

\begin{rawhtml}
<a href="http://www.techfak.uni-kiel.de/">
  <img border=0 alt="[Technical Faculty]" height=20  src="/images/tflogo.gif"></a>
<hr>
\end{rawhtml}


\maketitle{}


\section*{Pflichtenheft}
\label{sec:pflichtenheft}

Die folgende Liste enth�lt das Pflichtenheft. Der erste Eintrag ist
jeweils der \emph{aktuelle.}


\begin{center}
  \begin{tabular}[t]{r@{\quad}l@{\quad\quad}l@{\quad\quad}p{9cm}}
    1.
    & 
    \url{pflichtenheft1/}{Version 1}
    &
    10.\ April 2002
    & 
    Originalversion zu Beginn
  \end{tabular}
\end{center}




\section*{Zeitplan}
\label{sec:zeitplan}


Die Dauer des Projektes ist durch die Dauer des Sommersemesters 2002
festgelegt: Beginn ist der 23.\ April, Ende der 27.\ Juli. Das sind
rechnerisch \emph{13.\ Wochen.} Tabelle~\ref{tab:semesteruebersicht} stellt
die geplanten Termine und Ziele f�r das Projekt zusammen.

\begin{table}[htbp]
  \begin{center}
    \begin{tabular}{rrp{9cm}}
      Woche & Datum & Ziel 
      \\\hline
      1. & 16.\ April  &  erste Grobspezifikation; 
      technische Rahmenbedingungen fertig
      \\
      2. & 23.\ April  &  erste Grobspezifikation; erste Besprechung;
      cvs-Repositorium fertig, Handouts durchgearbeitet, Einteilung
      \\
      3. & 30.\ April &
      abstrakte Syntax funktionsf�hig
      \\
      4. &  7.\ Mai & klar(ere) Vorstellungen der gew�hlten Aufgabe,
      Gemeinschaftsbesprechung  der Schnittstellen, eventl.\ Vorstellungen
      �ber \Java-Spezifika der Library
      \\
      5.  & 14.\ Mai  & erster (nicht funktionsf�higer)
      Integrations/Kompilationstest (``stubs'') 
      \textbf{konkrete Schnittstellen (Java!)} vorhanden
      \\
      ??. & ?? & Abschlu�, Review-Treffen, Schlu�demo
    \end{tabular}
    \caption{Semester�bersicht}
    \label{tab:semesteruebersicht}
  \end{center}
\end{table}

Der wirklich feste Termin ist das Semesterende. Wichtiger noch als das
sklavische Einhalten der sich vorgenommenen Termine und Ziele ist
gegebenenfalls, rechtzeitig zu erkennen, da� ein Ziel sich als
unrealistisch herausstellt und dies auch in den Besprechungen offen,
begr�ndet und realistisch\footnote{Ein oft geh�rtes Symptom einer
  unrealistischen Lageeinsch�tzung geht ungef�hr so: ``Wir sind die
  letzten drei Wochen nicht viel weiter gekommen, aber das ist kein
  Problem, denn von nun an werden wir dreimal so schnell arbeiten.''}  zur
Sprache zu bringen, so da� man geeignet und rechtzeitig darauf reagieren
kann, z.B., durch Redefinition der Ziele, Umlagern der Last oder
�hnlichem. Wichtiger und am Ende des Projektes befriedigender, als viele
Features ein bischen und wackelig realisiert zu haben ist es, eine
sinnvolle, geringere Auswahl vollst�ndig und sicher implementiert zu
haben.








%%% Local Variables: 
%%% mode: latex
%%% TeX-master: "main"
%%% End: 


\section*{Arbeitsgruppen}
\label{sec:gruppen}

Jedes der Teams bearbeitet i.d.R.\ ein Java-Paket. Die Beschreibung der
Aufgabenstellung findet sich im Pflichtenheft.

\iffalse
\begin{table}[htbp]
  \centering
  \begin{tabular}[t]{l@{\quad\quad}l}
     Paket  &  Gruppe
     \\\hline
     Gui/Integration & 
     \\
     Editor &
     \\
     Checks & 
     \\
     SMV-Backend
     &
     \\
     Simulator
     &
  \end{tabular}
  \caption{Gruppeneinteilung}
  \label{tab:gruppen}
\end{table}
\fi



Die \emph{Mailadressen} der Teilnehmner finden sich (intern) unter
\begin{verbatim}
      $WORKDIR/Slime/org/gruppen.txt
\end{verbatim}


\section*{Baselines}
\label{sec:Baselines}


Die \emph{Baselines} sind hier zur schnelleren Referenz als \emph{Javaarchiv}
\texttt{slime\_v[x].jar} festgehalten, wobei \texttt{[x]} die Nummber des
Schnappschusses ist. Als Entwickler mit Zugriff auf das Quellcoderepositorium
kann man die Entwicklungsschritte sie nat�rlich mittels \cvs{} wieder
zur�ckholen. Zum Ausf�hren der Archivs speichere man die Datei
\texttt{slime\_v[x].jar} an einen geeigneten Platz, und rufe

\begin{verbatim}
  java -classpath [geeigneter_platz]/slime_v[x].jar gui.Gui
\end{verbatim}
auf; alternativ kann man auch fest
\begin{verbatim}
 export CLASSPATH=[geeigneter_platz]/slime_v[x].jar
\end{verbatim}
setzen, dann kann man sich die option \texttt{-classpath} schenken.


\medskip

\begin{tabular}{llll}
  3. & \url{baselines/slime\_v3.0.jar}{Baseline 3}
  &
  Parser an Bord
  &
  1. Juli 2001
  \\
  2. & \url{baselines/snot\_v2.0.jar}{Baseline 2}
  &
  Simulator an Bord
  &
  13. Juni 2001
  \\
  1. & \url{baselines/snot\_v1.0.jar}{Baseline 1}
  &
  erste Gesamtkompilation
  &
  5. Juni 2001
\end{tabular}

\section*{HTML-Dokumentation}
\label{sec:html-doc}



Zur schnelleren Orientierung wird die mit \javadoc{} dokumentierte
Implementierung im Netz bereitgestellt. Die Seiten werden bei gr��eren
Entwicklungsschritten aufgefrischt.

\begin{center}
  \importantxx{Dokumentation \url{doc}{Slime}}: momentan (\today): 
  momentan leer
\end{center}









\section*{Conventions and rules of the game}
\label{sec:conventions}

Besides the \emph{CVS-strategies} which have been handed out and discussed,
we should take care of the following:

\begin{itemize}
\item \textbf{Makefile}: each package, i.e., the root of the respective
  sub-directory must containt a \texttt{Makefile}.  Its first target must
  be \texttt{make all}, which generates for the packges the \Java{} byte
  code (whatever it takes in intermediate steps).  Additionally, a target
  \texttt{make clean} must be supported, which removes the byte-code and
  all temporary files again.  An example for a workable Makefile is
  contained in the \texttt{absynt}-package which covers the (\emph{very
    simple}) needs just described.

  
\item \textbf{Documentation:} the \Java-code should be meaningfully
  commented and documented. Information about the implementations which are
  relevant for the co-developers, are done in \javadoc. This especially
  applies to the intended meaning and usage of the public interfaces
  between the packages. Part of the documentation should be the name of the
  developer(s) and the version of the file. Further useful information
  could be the status of the method, class, \ldots, e.g., whether the
  implementation currently is only available as stub, whether the
  functionality is finished, but not yet tested, which are the known
  bugs/features \ldots.
  
  The \javadoc-comments may serve the internal communication to state what
  the package offers (and what it assumes). The generated web-pages are
  available via the project page, they are refreshed from time to time, and
  kept
  \url{http://www.informatik.uni-kiel.de/inf/deRoever/SS02/Java/Slime/doc}{here}.
\end{itemize}



%%%%%%%%%%%%%%%%%%%%%%%%%%%%%%%%%%%%%%%%%%%%%%%%%%%%%%%%%%%%
%% $Id: konventionen.tex,v 1.3 2002-07-10 09:30:20 swprakt Exp $
%%%%%%%%%%%%%%%%%%%%%%%%%%%%%%%%%%%%%%%%%%%%%%%%%%%%%%%%%%%%
%%% Local Variables: 
%%% mode: latex
%%% TeX-master: "main"
%%% End: 




\bibliographystyle{alpha}



\bibliography{string,etc,oop,crossref}


\end{document}



%%% Local Variables: 
%%% mode: latex
%%% TeX-master: t
%%% End: 

