


\newif\ifweb\webtrue    %%%%%% htmlonly geht irgendwie nicht
\newif\ifsolution\solutionfalse
\newif\ifworking\workingfalse


%%% Local Variables: 
%%% mode: latex
%%% TeX-master: "main"
%%% End: 


\documentclass[11pt,german]{article}
\usepackage{hevea}
%\usepackage{url}
\usepackage{babel}
\usepackage{epsfig}
\usepackage{tfheader}
\usepackage{a4wide}
\usepackage[latin1]{inputenc}

%\newcommand{\kommentar}[1]{[{\small\em #1}]

\newcommand{\Slime}{\textsc{Slime}}
\newcommand{\cvs}{cvs}
\newcommand{\Java}{\textsc{Java}}
\newcommand{\SMV}{\textsc{Smv}}
\newcommand{\Cplusplus}{C$^{++}$}
\newcommand{\javadoc}{\textsc{javadoc}}

\newcommand{\Int}    {\mathit{Int}}
\newcommand{\Bool}   {\mathit{Bool}}


\newcommand{\team}[1]{\textbf{Responsible:} #1\bigskip{}}



\newcommand{\bnfdef}{::=}
\newcommand{\bnfbar}{\ensuremath{\mid}}
\newcommand{\of}    {\ensuremath{\mathrel{:}}}
\newcommand{\without}{\backslash}
\newcommand{\ltrue}   {\mathit{true}}
\newcommand{\lfalse}  {\mathit{false}}

\newcommand{\suchthat}{\mathrel{\mid}}
\newcommand{\union}  {\cup}
\newcommand{\intersect}  {\cap}
\newcommand{\sizeof}[1]  {{\mid}{#1}{\mid}}
\newcommand{\sem}    [2]        {[\![#2]\!]_{#1}}      %semantics

\newenvironment{diagram}{\begin{displaymath}}{\end{displaymath}}

%\newcommand{\inputcode}[2][Code]   {
%  {\small
%  \mbox{}
%  \newline
%  \mbox{}
%  \hrulefill
%  \verbatiminput{#1/#2.java}
%  \hrulefill}}




\newcommand{\inputcode}[2][.]{
  {\footnotesize
  \mbox{}
  \newline\nopagebreak{}
  \mbox{}
  \hrulefill
  \lstinputlisting{#1/#2}
  \hrulefill}}

\newenvironment{code}{%
%  \small\mbox{}\nopagebreak{}\mbox\hrulefill{}
  {\begin{lstlisting}{}}}{%
  {\end{lstlisting}}}





%%%%%%%%%%%%%%%%%%%%%%%%%%%%%%%%%%%%%%% macros for semantics of SFC's

%%% RULES 


\newenvironment{ruleset}{
  \mbox{}\noindent\hrulefill\begin{displaymath}
    \renewcommand{\arraystretch}{1.6}\begin{array}[b]{l}}{
  \end{array}\end{displaymath}\hrulefill\mbox{}}




\newcommand{\treefont}{\small}
\newcommand{\leaf}[1]{{\begin{array}{c} #1 \end{array}}}
\newenvironment{proofleaf}{\begin{array}{c}}{\end{array}}

\newcommand{\namedruletree}[3]{{
     \treefont
     \prooftree 
       {\leaf{#2}}
     \justifies
       #3
     \using \rn{#1}
     \endprooftree
     }}                             

\newcommand{\ruletree}[2]{\namedruletree{}{#1}{#2}}


\newcommand{\rn}[1]{\mbox{\textsc{#1}}}   %% determines the font
\newcommand{\andalso}{\quad\quad}
\newcommand{\infrule}[3]{\namedruletree{#1}{#2}{#3}\andalso}
\newcommand{\scinfrule}[4]{\namedruletree{\mbox{$#4$}\quad\quad #1}{#2}{#3}\andalso}
\newcommand{\infax}[2]{
  \treefont
  #2\quad\quad \rn{#1}\andalso
  \andalso
  }
\newcommand{\scinfax}[3]{
  \treefont
  #2\quad \mbox{$#3$\quad\quad#1}
  \andalso
}




%%% SFCs
\newcommand{\init}          {\mathit{init}}
\newcommand{\act}           {\mathit{act}}
\newcommand{\Act}           {\mathit{Act}}
\newcommand{\States}        {\mathit{States}}
\newcommand{\Status}        {\mathit{Status}}
\newcommand{\Steps}         {\mathit{Steps}}
\newcommand{\Store}         {\mathit{Store}}
\newcommand{\Var}           {\mathit{Var}}
\newcommand{\Varin}         {\Var_{\mathit{in}}}
\newcommand{\Varout}        {\Var_{\mathit{out}}}
\newcommand{\Varloc}        {\Var_{\mathit{loc}}}

\newcommand{\src}           {\mathit{src}}
\newcommand{\target}        {\mathit{tar}}
\newcommand{\guard}         {\mathit{guard}}
\newcommand{\Expr}          {\mathit{Expr}}
\newcommand{\OF}            {\mathrel{:}}

\newcommand{\setto}[2]             {{\scriptstyle [#1\mathop{\mapsto}#2]}}  %setto{x}{v} : [x|->v]  {}
\newcommand{\substfor}[2]   {}

\newcommand{\inputstatus}       {\mathsf{I}}%{\mathit{WAITINPUT}}
\newcommand{\calcstatus}[1]     {\mathsf{C}(#1)}%{\mathit{DONEINPUT(#1)}}
\newcommand{\outputstatus}[1]   {\mathsf{O}(#1)}%{\mathit{FINISHED(#1)}}









%\newenvironment{diagram}{\begin{displaymath}}{\end{displaymath}}


\newcommand{\edge}[1]            {\longrightarrow_{#1}} %

%%% semantical steps
\newcommand{\trans}[1]             {\rightarrow_{#1}}
\newcommand{\transin}[1]           {\trans{?#1}}
\newcommand{\transout}[1]          {\trans{!#1}}








%%% Local Variables: 
%%% mode: latex
%%% TeX-master: t
%%% End: 



\title{{\huge\bf \textsl{S}equentia\textsl{l} Funct\textsl{i}on Charts
    \textsl{M}odeling \textsl{E}nvironment (aka.\ \Slime)}}
%\author{}
%\url{}{Martin Steffen}}
\date{}



\ifweb

\htmlfoot{\hrulefill{}
  {\footnotesize Pages last (re-)generated \today}}
\renewcommand{\@bodyargs}{bgcolor="white" alink="red" vlink="\#407999"  link="\#7070ff"}  
\fi




\begin{document}
\vspace{-2cm}


%\pagestyle{empty}

\begin{rawhtml}
<a href="http://www.techfak.uni-kiel.de/">
  <img border=0 alt="[Technical Faculty]" height=20  src="/images/tflogo.gif"></a>
<hr>
\end{rawhtml}


\maketitle{}


\section*{Pflichtenheft}
\label{sec:pflichtenheft}

Die folgende Liste enth�lt das Pflichtenheft. Der erste Eintrag ist
jeweils der \emph{aktuelle.}


\begin{center}
  \begin{tabular}[t]{r@{\quad}l@{\quad\quad}l@{\quad\quad}p{9cm}}
    1.
    & 
    \url{pflichtenheft1/}{Version 1}
    &
    10.\ April 2002
    & 
    Originalversion zu Beginn
  \end{tabular}
\end{center}




\section*{Zeitplan}
\label{sec:zeitplan}



Die Dauer des Projektes ist festgelegt durch die Dauer des Sommersemesters
2002: Beginn ist der 16.\ April, Ende der 17.\ Juli. Das sind rechnerisch
\emph{14.\ Praktikumstermine.} Tabelle~\ref{tab:semesteruebersicht} stellt
die geplanten Termine und Ziele f�r das Projekt zusammen.

\begin{table}[htbp]
  \begin{center}
    \begin{tabular}{rrp{9cm}}
      Woche & Datum & Ziel 
      \\\hline
      1. & 16.\ April  &  Vorstellung von SFC's + potentielle Pakete,
      Einteilung besprechen, Handouts,
      cvs-Repositorium fertig, Vorstellung CVS + Spielregeln
      \\
      2. & 23.\ April  &  Einteilung fest, Schnittstellenvorschlag vorstellen,
      abstrakte Syntax vorstellen, technische Rahmenbedingungen 
      funktionsf�hig (accounts, Zugriff, Java-Zusatzpakete \ldots)! 
      Zwischen der 2.\ und der 3.\ Woche: Einzel/Gruppengespr�che �ber die
      Pakete (z.B.\ auch zur Entscheidungshilfe)
      \\
      3. & 30.\ April & 
      Paketvorstellung durch die Teilnehmer (schriftlicher Zeitplan,
      Features, erwartete Probleme, �berschneidungen/Interaktion mit
      anderen Paketen \ldots)
      \\
      & 12.\ Juni &
      Erste Gesamtkompilation
      \\
       & \textbf{18.\ Juli} & Abschlu�, Review-Treffen, Schlu�demo
    \end{tabular}
    \caption{Semester�bersicht}
    \label{tab:semesteruebersicht}
  \end{center}
\end{table}

Die zu erbringende Leistung besteht nicht nur aus dem Paket alleine,
sondern auch die aktive Projektteilnahme. Das beinhaltet insbesondere
\emph{w�chentlicher Fortschrittsbericht jeder Gruppe, d.h.\ explizite
  Stellungnahme z.B.\ zu folgenden Punkten}
\begin{itemize}
\item gegenw�rtiger Status, insbesondere in Hinblick auf den eigenen
  Zeitplan
\item aufgetretene Probleme
\item eventuellen Verz�gerungen
\item (begr�ndete) W�nsche an andere Gruppen
\item Stellungnahme zu aufgetretenen Fehlern im eigenen Teil  \ldots
\end{itemize}




Ein fester Termin ist das Semesterende.  Es ist \emph{wichtig,} rechtzeitig
zu erkennen, da� ein Ziel sich als unrealistisch herausstellt und dies auch
in den Besprechungen offen, begr�ndet und realistisch\footnote{Ein oft
  geh�rtes Symptom einer unrealistischen Lageeinsch�tzung geht ungef�hr so:
  ``Wir sind die letzten drei Wochen nicht viel weiter gekommen, aber das
  ist kein Problem, denn von nun an werden wir dreimal so schnell
  arbeiten.''}  zur Sprache zu bringen, so da� man geeignet und rechtzeitig
darauf reagieren kann, z.B., durch Redefinition der Ziele, Umlagern der
Last oder �hnlichem. Wichtiger und am Ende des Projektes befriedigender,
als viele Features ein bischen und wackelig realisiert zu haben ist es,
eine sinnvolle, geringere Auswahl vollst�ndig und sicher implementiert zu
haben.


Es ist ebenfalls \emph{wichtig,} sich nicht nur f�r sein eigenes Paket
verantwortlich zu f�hlen, sondern ebenfalls f�r das Gesamtprojekt, d.h.,
man soll die von den anderen Gruppen im eigenen Pakete entdeckten und
reportierten Fehler ernst nehmen (die Fehler, Features etc.\ werden
festgehalten, und umgekehrt sich nicht scheuen, Fehler in anderen Paketen
zu reportieren. Wir werden eine \emph{mailingliste} zum Zweck der
Koordination einrichten. Die Adresse lautet
\begin{center}
  \url{mailto:swprakt+slime@informatik.uni-kiel.de}{swprakt+slime@informatik.uni-kiel.de}
\end{center}






%%% Local Variables: 
%%% mode: latex
%%% TeX-master: "main"
%%% End: 


\section*{Arbeitsgruppen}
\label{sec:gruppen}

Jedes der Teams bearbeitet i.d.R.\ ein Java-Paket. Die Beschreibung der
Aufgabenstellung findet sich im Pflichtenheft.

\iffalse
\begin{table}[htbp]
  \centering
  \begin{tabular}[t]{l@{\quad\quad}l}
     Paket  &  Gruppe
     \\\hline
     Gui/Integration & 
     \\
     Editor &
     \\
     Checks & 
     \\
     SMV-Backend
     &
     \\
     Simulator
     &
  \end{tabular}
  \caption{Gruppeneinteilung}
  \label{tab:gruppen}
\end{table}
\fi



Die \emph{Mailadressen} der Teilnehmner finden sich (intern) unter
\begin{verbatim}
      $WORKDIR/Slime/org/gruppen.txt
\end{verbatim}


\section*{Baselines}
\label{sec:Baselines}


Die \emph{Baselines} oder \emph{Snapshots} sind hier zur schnelleren
Referenz als \emph{Javaarchiv} \texttt{slime\_v[x].jar} festgehalten, wobei
\texttt{[x]} die Nummber des Schnappschusses ist. Als Entwickler mit
Zugriff auf das Quellcoderepositorium kann man die Entwicklungsschritte sie
nat�rlich mittels \cvs{} wieder zur�ckholen. Zum Ausf�hren der Archivs
speichere man die Datei \texttt{slime\_v[x].jar} an einen geeigneten Platz,
und rufe

\begin{verbatim}
  java -classpath [geeigneter_platz]/slime_v[x].jar gui.Gui
\end{verbatim}
auf; alternativ kann man auch fest
\begin{verbatim}
 export CLASSPATH=[geeigneter_platz]/slime_v[x].jar
\end{verbatim}
setzen, dann kann man sich die option \texttt{-classpath} schenken.


\medskip

\iffalse
\begin{tabular}{llll}
  3. & \url{baselines/slime\_v3.0.jar}{Baseline 3}
  &
  Parser an Bord
  &
  1. Juli 2001
  \\
  2. & \url{baselines/snot\_v2.0.jar}{Baseline 2}
  &
  Simulator an Bord
  &
  13. Juni 2001
  \\
  1. & \url{baselines/snot\_v1.0.jar}{Baseline 1}
  &
  erste Gesamtkompilation
  &
  5. Juni 2001
\end{tabular}
\fi

\section*{HTML-Dokumentation}
\label{sec:html-doc}



Zur schnelleren Orientierung wird die mit \javadoc{} dokumentierte
Implementierung im Netz bereitgestellt. Die Seiten werden bei gr��eren
Entwicklungsschritten aufgefrischt.

\begin{center}
  \importantxx{Dokumentation \url{doc}{Slime}}: momentan (\today): 
  momentan leer
\end{center}









\section*{Konventionen und Spielregeln}
\label{sec:konventionen}

Neben den gesondert ausgeteilten und diskutierten \emph{CVS-Spielregeln}
sollen folgenden Dinge beachtet werden.

\begin{itemize}
\item \textbf{Makefile}: jedes Paket, d.h.\ die Wurzel des entsprechenden
  Unterverzeichnisses soll ein \texttt{Makefile} enhalten. Als erstes Target
  mu� \texttt{make all} unterst�tzt werden, welche f�r das Paket den Java
  Bytecode erzeugt. Daneben soll ein \texttt{make clean} unterst�zt werden,
  welches den byte-code und tempor�re Dateien wieder entfernt. Ein einfaches
  Beispiel f�r ein passendes Makefile findet sich im Unterverzeichnis 
  \texttt{src/templates/}.
\item \textbf{Dokumentation:} Der \Java-Code soll sinnvoll kommentiert und
  dokumentiert werden.  F�r Information �ber die Implementierung, die f�r
  die Mit-Entwickler von Interesse ist, soll dies mittels \javadoc{}
  geschehen.  Dies betrifft insbesondere die Methoden, die zur Schnittstelle
  mit anderen Paketen geh�ren. Zur �ffentlichen (aber projekt-interne)
  Dokumentation geh�rt auch der Name der Entwickler und die Version.
  Weitere sinnvolle Information kann den \emph{Status} der Methode, Klasse,
  oder sonstigen Programmteils betreffen. Beispielsweise, ob die
  Implementierung noch virtuell ist (als stub), ob nur Teile der
  vereinbarten Funktionalit�t bereitstehen, ob die Funktionalit�t noch
  ungetested ist, ob Fehler bekannt sind eind.
  
  Die \javadoc-Kommentare dienen unter anderem zur internen Koordination,
  d.h.\ den anderen Gruppen Schnittstelle und Status des eigenen Paketes
  mitzuteilen.  Die generierten Seiten werden in regelm��igen Abst�nden in
  unserer
  \url{http://www.informatik.uni-kiel.de/inf/deRoever/SS01/Java/Snot/doc}{Projektseite}
  aktualisiert.
\end{itemize}


%%% Local Variables: 
%%% mode: latex
%%% TeX-master: "main"
%%% End: 




\bibliographystyle{alpha}



\bibliography{string,etc,oop,crossref}


\end{document}



%%% Local Variables: 
%%% mode: latex
%%% TeX-master: t
%%% End: 

