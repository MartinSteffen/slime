


\newif\ifweb\webtrue    %%%%%% htmlonly geht irgendwie nicht
\newif\ifsolution\solutionfalse
\newif\ifworking\workingfalse


%%% Local Variables: 
%%% mode: latex
%%% TeX-master: "main"
%%% End: 


\documentclass[11pt,german]{article}
\usepackage{hevea}
%\usepackage{url}
\usepackage{babel}
\usepackage{epsfig}
\usepackage{tfheader}
\usepackage{a4wide}
\usepackage[latin1]{inputenc}

%\newcommand{\kommentar}[1]{[{\small\em #1}]

\newcommand{\Slime}{\textsc{Slime}}
\newcommand{\cvs}{cvs}
\newcommand{\Java}{\textsc{Java}}
\newcommand{\SMV}{\textsc{Smv}}
\newcommand{\Cplusplus}{C$^{++}$}
\newcommand{\javadoc}{\textsc{javadoc}}

\newcommand{\Int}    {\mathit{Int}}
\newcommand{\Bool}   {\mathit{Bool}}


\newcommand{\team}[1]{\textbf{Responsible:} #1\bigskip{}}



\newcommand{\bnfdef}{::=}
\newcommand{\bnfbar}{\ensuremath{\mid}}
\newcommand{\of}    {\ensuremath{\mathrel{:}}}
\newcommand{\without}{\backslash}
\newcommand{\ltrue}   {\mathit{true}}
\newcommand{\lfalse}  {\mathit{false}}

\newcommand{\suchthat}{\mathrel{\mid}}
\newcommand{\union}  {\cup}
\newcommand{\intersect}  {\cap}
\newcommand{\sizeof}[1]  {{\mid}{#1}{\mid}}
\newcommand{\sem}    [2]        {[\![#2]\!]_{#1}}      %semantics

\newenvironment{diagram}{\begin{displaymath}}{\end{displaymath}}

%\newcommand{\inputcode}[2][Code]   {
%  {\small
%  \mbox{}
%  \newline
%  \mbox{}
%  \hrulefill
%  \verbatiminput{#1/#2.java}
%  \hrulefill}}




\newcommand{\inputcode}[2][.]{
  {\footnotesize
  \mbox{}
  \newline\nopagebreak{}
  \mbox{}
  \hrulefill
  \lstinputlisting{#1/#2}
  \hrulefill}}

\newenvironment{code}{%
%  \small\mbox{}\nopagebreak{}\mbox\hrulefill{}
  {\begin{lstlisting}{}}}{%
  {\end{lstlisting}}}





%%%%%%%%%%%%%%%%%%%%%%%%%%%%%%%%%%%%%%% macros for semantics of SFC's

%%% RULES 


\newenvironment{ruleset}{
  \mbox{}\noindent\hrulefill\begin{displaymath}
    \renewcommand{\arraystretch}{1.6}\begin{array}[b]{l}}{
  \end{array}\end{displaymath}\hrulefill\mbox{}}




\newcommand{\treefont}{\small}
\newcommand{\leaf}[1]{{\begin{array}{c} #1 \end{array}}}
\newenvironment{proofleaf}{\begin{array}{c}}{\end{array}}

\newcommand{\namedruletree}[3]{{
     \treefont
     \prooftree 
       {\leaf{#2}}
     \justifies
       #3
     \using \rn{#1}
     \endprooftree
     }}                             

\newcommand{\ruletree}[2]{\namedruletree{}{#1}{#2}}


\newcommand{\rn}[1]{\mbox{\textsc{#1}}}   %% determines the font
\newcommand{\andalso}{\quad\quad}
\newcommand{\infrule}[3]{\namedruletree{#1}{#2}{#3}\andalso}
\newcommand{\scinfrule}[4]{\namedruletree{\mbox{$#4$}\quad\quad #1}{#2}{#3}\andalso}
\newcommand{\infax}[2]{
  \treefont
  #2\quad\quad \rn{#1}\andalso
  \andalso
  }
\newcommand{\scinfax}[3]{
  \treefont
  #2\quad \mbox{$#3$\quad\quad#1}
  \andalso
}




%%% SFCs
\newcommand{\init}          {\mathit{init}}
\newcommand{\act}           {\mathit{act}}
\newcommand{\Act}           {\mathit{Act}}
\newcommand{\States}        {\mathit{States}}
\newcommand{\Status}        {\mathit{Status}}
\newcommand{\Steps}         {\mathit{Steps}}
\newcommand{\Store}         {\mathit{Store}}
\newcommand{\Var}           {\mathit{Var}}
\newcommand{\Varin}         {\Var_{\mathit{in}}}
\newcommand{\Varout}        {\Var_{\mathit{out}}}
\newcommand{\Varloc}        {\Var_{\mathit{loc}}}

\newcommand{\src}           {\mathit{src}}
\newcommand{\target}        {\mathit{tar}}
\newcommand{\guard}         {\mathit{guard}}
\newcommand{\Expr}          {\mathit{Expr}}
\newcommand{\OF}            {\mathrel{:}}

\newcommand{\setto}[2]             {{\scriptstyle [#1\mathop{\mapsto}#2]}}  %setto{x}{v} : [x|->v]  {}
\newcommand{\substfor}[2]   {}

\newcommand{\inputstatus}       {\mathsf{I}}%{\mathit{WAITINPUT}}
\newcommand{\calcstatus}[1]     {\mathsf{C}(#1)}%{\mathit{DONEINPUT(#1)}}
\newcommand{\outputstatus}[1]   {\mathsf{O}(#1)}%{\mathit{FINISHED(#1)}}









%\newenvironment{diagram}{\begin{displaymath}}{\end{displaymath}}


\newcommand{\edge}[1]            {\longrightarrow_{#1}} %

%%% semantical steps
\newcommand{\trans}[1]             {\rightarrow_{#1}}
\newcommand{\transin}[1]           {\trans{?#1}}
\newcommand{\transout}[1]          {\trans{!#1}}








%%% Local Variables: 
%%% mode: latex
%%% TeX-master: t
%%% End: 



\title{{\huge\bf \textsl{S}equentia\textsl{l} Funct\textsl{i}on Charts
    \textsl{M}odeling \textsl{E}nvironment (aka.\ \Slime)}}
%\author{}
%\url{}{Martin Steffen}}
\date{}



\ifweb

\htmlfoot{\hrulefill{}
  {\footnotesize Pages last (re-)generated \today}}
\renewcommand{\@bodyargs}{bgcolor="white" alink="red" vlink="\#407999"  link="\#7070ff"}  
\fi




\begin{document}
\vspace{-2cm}


%\pagestyle{empty}

\begin{rawhtml}
<a href="http://www.techfak.uni-kiel.de/">
  <img border=0 alt="[Technical Faculty]" height=20  src="/images/tflogo.gif"></a>
<hr>
\end{rawhtml}


\maketitle{}




\section*{Review}

Date of the final review, demo, and debriefing is fixed to

\begin{center}
  \importantxx{\textbf{18.07.2002, 17:00}} 
\end{center}

Responsible for the organization of the event is Norbert Boeck.



%%% Local Variables: 
%%% mode: latex
%%% TeX-master: "main"
%%% End: 



\section*{Java-doc}
\label{sec:javadoc}
\label{sec:html-doc}



For convenient lookup for those who like \javadoc, the public documentation
and comments are avaiable here. The pages are refreshed from time to time,
especially after larger development steps.

\begin{center}
  \importantxx{documentation \url{doc}{Slime}}: currently (\today): all
  packages on board, but currently non-compilable (cf.\ the jar'ed
  snapshots for runnable intermediate stages.)
\end{center}

\section*{Snapshots}
\label{sec:snapshots}


The \emph{baselines} or \emph{snapshots} are archived here for quick
reference as \texttt{slime\_v[x].jar}, where \texttt{[x]} is the number or
the tag of the snapshot. Developers with access to the source repository
can retrieve those snapshots of course with cvs.\footnote{The command is
  \texttt{cvs update -r [tag]}. Be careful with this command, don't
  generate development branches without knowing.}


To \emph{execute} one of the archived snapshots, save it at some convenient
place, set the classpath to point to the jar-archve, and invoke the
interpreter, i.e., Java's virtual machine as indicated in the table below
snapshot.

For instance, after downloading, you can do

\begin{verbatim}
  java -classpath ./slime_v[x].jar <command>} 

\end{verbatim}
to start the tool. Alternatively, you can set the
\texttt{CLASSPATH}-variable appropriatly, in which case the
\texttt{-classpath}-option is not needed.

\medskip


\begin{tabular}{llll}
  \\\hline
  &
  archive/date
  &
  explanation
  &
  command
  \\\hline
  2. & \url{baselines/snot\_vsecond.compilation-sep.jar}{9. July 2002}
  &
  second compilation (separately)
  &
  \texttt{java slime.editor.EditorInFrame}
  \\
%  1. & 
%  &
%  first compilation of all packages
%  &
%  3. July 2002
\end{tabular}




%%% Local Variables: 
%%% mode: latex
%%% TeX-master: "main"
%%% End: 



\section*{Pflichtenheft}
\label{sec:pflichtenheft}

Die folgende Liste enth�lt das Pflichtenheft. Der erste Eintrag ist
jeweils der \emph{aktuelle.}


\begin{center}
  \begin{tabular}[t]{r@{\quad}l@{\quad\quad}l@{\quad\quad}p{9cm}}
    4.
    & 
    \url{requirements4/}{Version 4}
    &
    10.\ July 2002
    & 
    sync' spec with the decisions/implementation
    \\
    3.
    & 
    \url{requirements3/}{Version 3}
    &
    14.\ Juni 2002
    & 
    kleinere Reparaturen an der Spezifikation
    \\
    2.
    & 
    \url{requirements2/}{Version 2}
    &
    26.\ April 2002
    & 
    Formale Semantik hinzugef�gt
    \\
    1.
    & 
    \url{requirements1/}{Version 1}
    &
    22.\ April 2002
    & 
    Originalversion zu Beginn
  \end{tabular}
\end{center}



%%% Local Variables: 
%%% mode: latex
%%% TeX-master: t
%%% End: 



\subsection*{Weekly meetings}
\label{sec:meetings}


This section contains the results of the weekly meetings are communicated
via email to the group members. There are kept here for quick reference.


\begin{center}
  \begin{tabular}[t]{r@{\quad}l@{\quad\quad}p{9cm}}
    0.
    &
    9. April
    &
    no meeting this week, only email
    \\
    1.
    &
    \url{meetings/meeting-160402.txt}{16. April}
    &
    task assignment, change of meeting time
    \\
    2.
    & 
    \url{meetings/meeting-240402.txt}{24. April}
    &
    no real decisions, preparation for next meeting
    \\
    3.
    &
    1.\ May
    &
    holyday
    \\
    4.
    &
    \url{meetings/meeting-080502.txt}{8.\ May}
    & 
    task presentation, status of editor group unclear,
    check group said ciao, restructuring planned
    \\
    5.
    &
    15.\ May
    &
    no email?
    \\
    6.
    &
    22. May
    &
    no email?
    \\
    7.
    &
    29.\ May
    &
    no email
    \\
    8.
    &
    \url{meetings/meeting-050602.txt}{5.\ June}
    &
    makeshift parser under utilites added; first
    integration set to 12.\ June
    \\
    9.
    &
    \url{meetings/meeting-120602.txt}{12.\ June}
    &
    restructuring now (freeze)!, makeshift checks will be implemented 
    (as visitor or otherwise), no integration (since code
    missing)
    \\
    10. 
    &
    \url{meetings/meeting-190602.txt}{19.\ June}
    &
    no integration yet
    \\
    11. 
    &
    \url{meetings/meeting-260602.txt}{26.\ June}
    &
    everyone on board; plan of final review;
    plan for last 3 weeks
    \\
    12. 
    &
    \url{meetings/meeting-030702.txt}{3.\ July}
    &
    removal of additional checked in stuff +
    removal of class files.
    \\
    13. 
    &
    \url{meetings/meeting-100702.txt}{10.\ July}
    &
    \\
    14. 
    &
    %\url{meetings/meeting-190602.txt}{17.\ July}
    17.\ July
    &
    \\
  \end{tabular}
\end{center}





Organisatorial/procedural things discussed during the meeting at 3.7.02
included arguments mentioned \url{slides/meeting030702.ps}{here.}


\iffalse
Here the \important{official decision} concerning the \texttt{CLASSPATH}
etc. (same is in \texttt{Readme.devel}).

\begin{itemize}
\item the \importantxx{checked-in} versions of cup and lex are
  obligatory\footnote{They replace the ones previously used under
    \texttt{/home/java}, which had been the official ones so far. In
    effect, they are the same, and just checked in.}
\item the following \importantxx{classpath} is obligatory:
  \texttt{CLASSPATH=<WORKDIR>/Slime/src:<WORKDIR>/Slime/lib/JLex.jar:<WORKDIR>/Slime/lib/java_cup.jar:}
  where of course the \texttt{<WORKDIR>} is a placeholder and it has to be
  adapted by the individual user to his work directory.
\item no \importantxx{generated} files nor \importantxx{class} files will be
  checked in (unless technical reasons call for it, in which case we will
  discuss this in the light of the new arguments again)
\item no \importantxx{non-slime} code or data will be checked in under
  \texttt{CLASSPATH=<WORKDIR>/Slime/src}
\item revisions \emph{log}s are useful and worth reading, but it's not
  mandatory to read each other's logs.
\end{itemize}
\fi


%%% Local Variables: 
%%% mode: latex
%%% TeX-master: "meetings"
%%% End: 




\section*{Zeitplan}
\label{sec:zeitplan}



Die Dauer des Projektes ist festgelegt durch die Dauer des Sommersemesters
2002: Beginn ist der 16.\ April, Ende der 17.\ Juli. Das sind rechnerisch
\emph{14.\ Praktikumstermine.} Tabelle~\ref{tab:semesteruebersicht} stellt
die geplanten Termine und Ziele f�r das Projekt zusammen.

\begin{table}[htbp]
  \begin{center}
    \begin{tabular}{rrp{9cm}}
      Woche & Datum & Ziel 
      \\\hline
      1. & 16.\ April  &  Vorstellung von SFC's + potentielle Pakete,
      Einteilung besprechen, Handouts,
      cvs-Repositorium fertig, Vorstellung CVS + Spielregeln
      \\
      2. & 23.\ April  &  Einteilung fest, Schnittstellenvorschlag vorstellen,
      abstrakte Syntax vorstellen, technische Rahmenbedingungen 
      funktionsf�hig (accounts, Zugriff, Java-Zusatzpakete \ldots)! 
      Zwischen der 2.\ und der 3.\ Woche: Einzel/Gruppengespr�che �ber die
      Pakete (z.B.\ auch zur Entscheidungshilfe)
      \\
      3. & 30.\ April & 
      Paketvorstellung durch die Teilnehmer (schriftlicher Zeitplan,
      Features, erwartete Probleme, �berschneidungen/Interaktion mit
      anderen Paketen \ldots)
      \\
      & 12.\ Juni &
      Erste Gesamtkompilation
      \\
       & \textbf{18.\ Juli} & Abschlu�, Review-Treffen, Schlu�demo
    \end{tabular}
    \caption{Semester�bersicht}
    \label{tab:semesteruebersicht}
  \end{center}
\end{table}

Die zu erbringende Leistung besteht nicht nur aus dem Paket alleine,
sondern auch die aktive Projektteilnahme. Das beinhaltet insbesondere
\emph{w�chentlicher Fortschrittsbericht jeder Gruppe, d.h.\ explizite
  Stellungnahme z.B.\ zu folgenden Punkten}
\begin{itemize}
\item gegenw�rtiger Status, insbesondere in Hinblick auf den eigenen
  Zeitplan
\item aufgetretene Probleme
\item eventuellen Verz�gerungen
\item (begr�ndete) W�nsche an andere Gruppen
\item Stellungnahme zu aufgetretenen Fehlern im eigenen Teil  \ldots
\end{itemize}




Ein fester Termin ist das Semesterende.  Es ist \emph{wichtig,} rechtzeitig
zu erkennen, da� ein Ziel sich als unrealistisch herausstellt und dies auch
in den Besprechungen offen, begr�ndet und realistisch\footnote{Ein oft
  geh�rtes Symptom einer unrealistischen Lageeinsch�tzung geht ungef�hr so:
  ``Wir sind die letzten drei Wochen nicht viel weiter gekommen, aber das
  ist kein Problem, denn von nun an werden wir dreimal so schnell
  arbeiten.''}  zur Sprache zu bringen, so da� man geeignet und rechtzeitig
darauf reagieren kann, z.B., durch Redefinition der Ziele, Umlagern der
Last oder �hnlichem. Wichtiger und am Ende des Projektes befriedigender,
als viele Features ein bischen und wackelig realisiert zu haben ist es,
eine sinnvolle, geringere Auswahl vollst�ndig und sicher implementiert zu
haben.


Es ist ebenfalls \emph{wichtig,} sich nicht nur f�r sein eigenes Paket
verantwortlich zu f�hlen, sondern ebenfalls f�r das Gesamtprojekt, d.h.,
man soll die von den anderen Gruppen im eigenen Pakete entdeckten und
reportierten Fehler ernst nehmen (die Fehler, Features etc.\ werden
festgehalten, und umgekehrt sich nicht scheuen, Fehler in anderen Paketen
zu reportieren. Wir werden eine \emph{mailingliste} zum Zweck der
Koordination einrichten. Die Adresse lautet
\begin{center}
  \url{mailto:swprakt+slime@informatik.uni-kiel.de}{swprakt+slime@informatik.uni-kiel.de}
\end{center}






%%% Local Variables: 
%%% mode: latex
%%% TeX-master: "main"
%%% End: 










\section*{Arbeitsgruppen}
\label{sec:gruppen}

Jedes der Teams bearbeitet i.d.R.\ ein Java-Paket. Die Beschreibung der
Aufgabenstellung findet sich im Pflichtenheft.

\begin{table}[htbp]
  \centering
  \begin{tabular}[t]{l@{\quad\quad}l}
     Paket  &  Verantwortlich
     \\\hline
     Gui/integration & Norbert Boeck
     \\
     editor & Andreas Niemann
     \\
     checks & all
     \\
     simulator & Immo Grabe
     \\
     parser & Marco Wendel,
     \\
     layout & Andreas Niemann
     \\
     abstract syntax & all
  \end{tabular}
  \caption{Gruppeneinteilung}
  \label{tab:gruppen}
\end{table}



Die \emph{Mailadressen} der Teilnehmer finden sich (intern) unter
\begin{verbatim}
      $WORKDIR/Slime/org/packages.txt
\end{verbatim}
Man kann auch eine email an \mathtt{swprakt+slime@informatik.uni-kiel.de}
schicken, die dann weiterverteilt wird.











\section*{Konventionen und Spielregeln}
\label{sec:konventionen}

Neben den gesondert ausgeteilten und diskutierten \emph{CVS-Spielregeln}
sollen folgenden Dinge beachtet werden.

\begin{itemize}
\item \textbf{Makefile}: jedes Paket, d.h.\ die Wurzel des entsprechenden
  Unterverzeichnisses soll ein \texttt{Makefile} enhalten. Als erstes Target
  mu� \texttt{make all} unterst�tzt werden, welche f�r das Paket den Java
  Bytecode erzeugt. Daneben soll ein \texttt{make clean} unterst�zt werden,
  welches den byte-code und tempor�re Dateien wieder entfernt. Ein einfaches
  Beispiel f�r ein passendes Makefile findet sich im Unterverzeichnis 
  \texttt{src/templates/}.
\item \textbf{Dokumentation:} Der \Java-Code soll sinnvoll kommentiert und
  dokumentiert werden.  F�r Information �ber die Implementierung, die f�r
  die Mit-Entwickler von Interesse ist, soll dies mittels \javadoc{}
  geschehen.  Dies betrifft insbesondere die Methoden, die zur Schnittstelle
  mit anderen Paketen geh�ren. Zur �ffentlichen (aber projekt-interne)
  Dokumentation geh�rt auch der Name der Entwickler und die Version.
  Weitere sinnvolle Information kann den \emph{Status} der Methode, Klasse,
  oder sonstigen Programmteils betreffen. Beispielsweise, ob die
  Implementierung noch virtuell ist (als stub), ob nur Teile der
  vereinbarten Funktionalit�t bereitstehen, ob die Funktionalit�t noch
  ungetested ist, ob Fehler bekannt sind eind.
  
  Die \javadoc-Kommentare dienen unter anderem zur internen Koordination,
  d.h.\ den anderen Gruppen Schnittstelle und Status des eigenen Paketes
  mitzuteilen.  Die generierten Seiten werden in regelm��igen Abst�nden in
  unserer
  \url{http://www.informatik.uni-kiel.de/inf/deRoever/SS01/Java/Snot/doc}{Projektseite}
  aktualisiert.
\end{itemize}


%%% Local Variables: 
%%% mode: latex
%%% TeX-master: "main"
%%% End: 




\bibliographystyle{alpha}



\bibliography{string,etc,oop,crossref}


\end{document}


%%% Local Variables: 
%%% mode: latex
%%% TeX-master: t
%%% End: 

