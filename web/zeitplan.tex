
\section*{Zeitplan}
\label{sec:zeitplan}



Die Dauer des Projektes ist festgelegt durch die Dauer des Sommersemesters
2002: Beginn ist der 16.\ April, Ende der 17.\ Juli. Das sind rechnerisch
\emph{14.\ Praktikumstermine.} Tabelle~\ref{tab:semesteruebersicht} stellt
die geplanten Termine und Ziele f�r das Projekt zusammen.

\begin{table}[htbp]
  \begin{center}
    \begin{tabular}{rrp{9cm}}
      Woche & Datum & Ziel 
      \\\hline
      1. & 16.\ April  &  Vorstellung von SFC's + potentielle Pakete,
      Einteilung besprechen, Handouts,
      cvs-Repositorium fertig, Vorstellung CVS + Spielregeln
      \\
      2. & 23.\ April  &  Einteilung fest, Schnittstellenvorschlag vorstellen,
      abstrakte Syntax vorstellen, technische Rahmenbedingungen 
      funktionsf�hig (accounts, Zugriff, Java-Zusatzpakete \ldots)! 
      Zwischen der 2.\ und der 3.\ Woche: Einzel/Gruppengespr�che �ber die
      Pakete (z.B.\ auch zur Entscheidungshilfe)
      \\
      3. & 30.\ April & 
      Paketvorstellung durch die Teilnehmer (schriftlicher Zeitplan,
      Features, erwartete Probleme, �berschneidungen/Interaktion mit
      anderen Paketen \ldots)
      \\
      & 12.\ Juni &
      Erste Gesamtkompilation
      \\
      14. & 23.\ Juli & Abschlu�, Review-Treffen, Schlu�demo
    \end{tabular}
    \caption{Semester�bersicht}
    \label{tab:semesteruebersicht}
  \end{center}
\end{table}

Die zu erbringende Leistung besteht nicht nur aus dem Paket alleine,
sondern auch die aktive Projektteilnahme. Das beinhaltet insbesondere
\emph{w�chentlicher Fortschrittsbericht jeder Gruppe, d.h.\ explizite
  Stellungnahme z.B.\ zu folgenden Punkten}
\begin{itemize}
\item gegenw�rtiger Status, insbesondere in Hinblick auf den eigenen
  Zeitplan
\item aufgetretene Probleme
\item eventuellen Verz�gerungen
\item (begr�ndete) W�nsche an andere Gruppen
\item Stellungnahme zu aufgetretenen Fehlern im eigenen Teil  \ldots
\end{itemize}




Ein fester Termin ist das Semesterende.  Es ist \emph{wichtig,} rechtzeitig
zu erkennen, da� ein Ziel sich als unrealistisch herausstellt und dies auch
in den Besprechungen offen, begr�ndet und realistisch\footnote{Ein oft
  geh�rtes Symptom einer unrealistischen Lageeinsch�tzung geht ungef�hr so:
  ``Wir sind die letzten drei Wochen nicht viel weiter gekommen, aber das
  ist kein Problem, denn von nun an werden wir dreimal so schnell
  arbeiten.''}  zur Sprache zu bringen, so da� man geeignet und rechtzeitig
darauf reagieren kann, z.B., durch Redefinition der Ziele, Umlagern der
Last oder �hnlichem. Wichtiger und am Ende des Projektes befriedigender,
als viele Features ein bischen und wackelig realisiert zu haben ist es,
eine sinnvolle, geringere Auswahl vollst�ndig und sicher implementiert zu
haben.


Es ist ebenfalls \emph{wichtig,} sich nicht nur f�r sein eigenes Paket
verantwortlich zu f�hlen, sondern ebenfalls f�r das Gesamtprojekt, d.h.,
man soll die von den anderen Gruppen im eigenen Pakete entdeckten und
reportierten Fehler ernst nehmen (die Fehler, Features etc.\ werden
festgehalten, und umgekehrt sich nicht scheuen, Fehler in anderen Paketen
zu reportieren. Wir werden eine \emph{mailingliste} zum Zweck der
Koordination einrichten. Die Adresse lautet
\begin{center}
  \url{mailto:swprakt+slime@informatik.uni-kiel.de}{swprakt+slime@informatik.uni-kiel.de}
\end{center}






%%% Local Variables: 
%%% mode: latex
%%% TeX-master: "main"
%%% End: 
