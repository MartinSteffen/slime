
\section*{Time line}
\label{sec:timeline}



The project's duration is determined by the length of the summer term 2002.
Start is April, 10th, the end is July, 17th. This gives nominally 14 weeks.
Table~\ref{tab:semester} shows a overview over the planned milestones of
the project.

\begin{table}[htbp]
  \begin{center}
    \begin{tabular}{rrp{9cm}}
      week & date & goal
      \\\hline
      1. & 16.\ April  &  presentation of sfc's, potential task and packages,
      discussion of assignments, handouts, cvs-repos ready, presentation
      of cvs + development stratety
      \\
      2. & 23.\ April  &  fixed task assignmens, proposal for interfaces,
      abstract syntax finished; technical  nonsense solved (account,
      access, additional java packages \ldots).
      During the 2.\ and the 3.\ week: separate meetings or extra subgroup
      meetings to clarify tasks and goals (e.g.\ also  for decision finding)
      \\
      3. & 30.\ April & 
      presentation of the packages by the developers (written plan with
      milestones expected problems, overlap/interaction with other packages
      \ldots)
      \\
      & 12.\ June &
      first common compilation
      \\
       & \textbf{18.\ Juli} & final review, demo, debriefing
    \end{tabular}
    \caption{Semester}
    \label{tab:semester}
  \end{center}
\end{table}


\begin{rawhtml}
  <center>
  <A HREF="timeline.png"><IMG  WIDTH=400 SRC="timeline.png"> </a>
  </center>
\end{rawhtml}



The work in the course consists not only in producing code!  Required is
also \emph{active participation} in the project, especially a weekly
\emph{progress report} of each groups, explicitely addressing the following
points:
\begin{itemize}
\item current status, especially wrt.\ one's one time plan
\item encountered or forseen problems
\item potential delays
\item (justified) requests for additions/changes \ldots to other groups
\item to comment on errors reported in one's own package, how they will be
  dealt with/have been dealt with, \ldots
\end{itemize}



A fixed date is the end of semester.  It is \emph{important} to realize in
time if a goal turns out to be unrealistically ambitious or if other
problems interfere with the goal. Furthermore it is important to addresse
any of those problems or delays in the working meetings in an \emph{open}
manner, to asses the situation realistically and explain real or expected
problems.\footnote{An often heard symptom of an unrealistic assessment of
  the situation goes like this: ``last 3 weeks, there was no progress, but
  that's not a problem, since I'm able to work 3 times as fast from now
  on.''} Only in this way it is possible to \emph{react} in time, for
instance by redefinition of the goal, re-assignment of a task, cutting down
the features \ldots.  Problems of this kind don't disappear by delaying
counter measures, and trying not to mention problems leads to delay.

In the end, it is more important and satisfying to implement a
well-defined, small set of features in a solid way than to incoporate tons
of things which probably would work with additional one more month of time,
or two.



Furthermore it is \emph{important,} not to feel responsible for one's one
package in isolation, but also for the group project, i.e., it's not a
shame report on other packages' errors and, conversely, to react upon
errors or purported errors reported by others. The error reports are kept
in a separte file. For importantant announcements, error reports, or
coordination messages, we have set up an email-list. It's address is
\begin{center}
  \url{mailto:swprakt+slime@informatik.uni-kiel.de}{swprakt+slime@informatik.uni-kiel.de}
\end{center}



%%%%%%%%%%%%%%%%%%%%%%%%%%%%%%%%%%%%%%%%%%%%%%%%%%%%%%%%%%%%
%% $Id: zeitplan.tex,v 1.12 2002-07-12 06:23:13 swprakt Exp $
%%%%%%%%%%%%%%%%%%%%%%%%%%%%%%%%%%%%%%%%%%%%%%%%%%%%%%%%%%%%


%%% Local Variables: 
%%% mode: latex
%%% TeX-master: "main"
%%% End: 
