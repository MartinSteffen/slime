\documentclass[a4paper,dvips]{foils}

%\def\printlandscape{\special{landscape}}

\usepackage{amsmath}
\usepackage{../inputs/ks-foils,../inputs/ks-symbols}
%\usepackage{folien}
\usepackage[latin1]{inputenc}


% Color defines
\definecolor{lightgray}{gray}{0.9}
\definecolor{darkgreen}{rgb}{0.0,0.4,0.0}
\definecolor{darkpurple}{rgb}{0.5,0.0,0.5}
\definecolor{orange}{rgb}{0.8,0.2,0.0}
\definecolor{darkblue}{rgb}{0.0,0.0,0.6}
\definecolor{darkred}{rgb}{0.7,0.0,0.0}
\definecolor{lightblue}{rgb}{0.6,0.6,1.0}

% Coloring macros
\newcommand{\IMPORTANTcolor}[1]{\textcolor{darkred}{#1}}
\newcommand{\STEPcolor}[1]{\textcolor{lightgray}{#1}}
%\newcommand{\STEPcolor}[1]{#1}
\newcommand{\GUARDcolor}[1]{\textcolor{darkgreen}{#1}}
\newcommand{\SPECcolor}[1]{\textcolor{darkpurple}{#1}}
\newcommand{\ABcolor}[1]{\textcolor{lightblue}{#1}}
\newcommand{\colorSPEC}{\color{darkpurple}}
\newcommand{\RESULTcolor}[1]{\textcolor{orange}{#1}}
\newcommand{\HEADBGcolor}[1]{\textcolor{blue}{#1}}
\newcommand{\HEADFGcolor}[1]{\textcolor{white}{#1}}
\newcommand{\TITLEcolor}[1]{\textcolor{darkblue}{\textbf{#1}}}


%%%%%%%%%%%%%%%%%%%%%%%%%%%%%%%%%%%%%%%%%%%%%%%%%%%%%%%%%%%%%%%%%%%%%%%%%%%%%%


% Macros for SFC pictures

\newcommand{\GUARD}[1][]{%
 \put(-1.5,0){\linethickness{0.2\unitlength}\line(1,0){3}}%
              \put(2.5,0){\makebox(0,0)[l]{\GUARDcolor{#1}}}}
\newcommand{\STEP}[1][]{{%
 \put(-2.5,-2.5){\STEPcolor{\rule{5\unitlength}{5\unitlength}}}%
 \put(-2.5,-2.5){\linethickness{0.2\unitlength}%
                 \framebox(5,5){#1}}}}
\newcommand{\STEPinitial}[1][]{{%
 \put(-2.5,-2.5){\STEPcolor{\rule{5\unitlength}{5\unitlength}}}%
 \put(-2.5,-2.5){\linethickness{0.2\unitlength}%
                 \framebox(5,5){#1}}}
 \put(-2.2,-2.2){\linethickness{0.2\unitlength}%
                 \framebox(4.4,4.4){}}}
\newcommand{\LINE}{\linethickness{0.15\unitlength}\line}
\newcommand{\VECTOR}{\thicklines\linethickness{0.15\unitlength}\vector}

% Mark a step as active
\newcommand{\Mark}{\put(-3,-3){\linethickness{0.5\unitlength}%
                   \IMPORTANTcolor{\framebox(6,6){}}}}

% Action block
\newcommand{\AB}[3][\put(2.5,1){\line(1,0){2}}]{\linethickness{0.15\unitlength}
  #1
  \put(4.5,-0.5){\ABcolor{\rule{12\unitlength}{3\unitlength}}}
  \put(4.5,2.5){\line(1,0){12}}\put(4.5,-0.5){\line(1,0){12}}
  \put(4.5,-0.5){\line(0,1){3}}\put(7.0,-0.5){\line(0,1){3}}
  \put(16.5,-0.5){\line(0,1){3}}
  \put(5.75,1){\makebox(0,0){\small\texttt{#2}}}
  \put(11.75,1){\makebox(0,0){\small\textit{#3}}}
}

% Mark an action as active
\newcommand{\MarkA}{\put(7,-0.5){\linethickness{0.4\unitlength}%
                    \IMPORTANTcolor{\framebox(9.5,3){}}}}

% Action block with timing constraint
\newcommand{\ABt}[3][]{\linethickness{0.15\unitlength}
  \put(2.5,1){\line(1,0){2}}
  \put(4.5,-1){\ABcolor{\rule{14.5\unitlength}{4\unitlength}}}
  \put(4.5,3){\line(1,0){14.5}}\put(4.5,-1){\line(1,0){14.5}}
  \put(4.5,-1){\line(0,1){4}}\put(9.5,-1){\line(0,1){4}}
  \put(19,-1){\line(0,1){4}}
  \put(7,2.7){\makebox(0,0)[t]{\small\texttt{#2}}}
  \put(7,-0.8){\makebox(0,0)[b]{\tiny\textrm{#1}}}
  \put(14.25,1){\makebox(0,0){\small\textit{#3}}}
}

% Mark a timed action as active
\newcommand{\MarkAt}{\put(9.5,-1){\linethickness{0.4\unitlength}%
                     \IMPORTANTcolor{\framebox(9.5,4){}}}}



\begin{document}
\unitlength2.5mm%


\newcommand{\INSTEP}[1]{\parbox{3cm}{\texttt{\footnotesize~~~~%
        \begin{tabular}{|c|l|} \hline
          N & #1 \\ \hline
        \end{tabular}
}}}
\newcommand{\INGUARD}[1]{{\texttt{\footnotesize~~#1}}}


\Pfoilhead{Sequential Function Charts (SFC)}

\begin{itemize}
\item SFC's: eine von mehreren standardisierten Beschreibungssprachen f�r
  Mikrokontroller-Programme
\item IEEE-Standard
\item Sprachen werden zusammen verwendet, um Mikrokontroller zu programmieren
  (d.h. eine Funktion eines Programmes in Sprache A kann in Sprache B
  geschrieben werden)
\item Assembler-artig, Pascal-artig, usw.
\item von vielen gro�en Firmen benutzt
\item SFC's: graphisch
\item Petrinetz-artig
\end{itemize}


\clearpage


\Pfoilhead{SFC's (2)}


\begin{picture}(44.5,56)(-27,1.5)
  \put(  0, 55){\put(  0,-2.5){\LINE(0,-1){6}}
    \put(  0,-5.5){\GUARD{\INGUARD{true}}}
    \put(-12,-8.5){\LINE(1,0){24}}
    \put(-12,-9.5){\LINE(1,0){24}}
    \put(-12,-8.5){\LINE(0,-1){4}}
    \put( 12,-8.5){\LINE(0,-1){4}}
    \put( 2,0){\LINE(1,0){2}}
    \STEPinitial[$s_1$]{\INSTEP{x\,:=\,false}}
    }
  \put(-12, 40){\put(  0,-2.5){\LINE(0,-1){25}}
    \put(  0,-14.5){\GUARD{\INGUARD{x and y}}}
    \put( 2,0){\LINE(1,0){2}}
    \STEP[$s_2$]{\INSTEP{y\,:=\,x}}
    }
  % oberster im rechten zweig
  \put( 12, 40){\put(  0,-2.5){\LINE(0,-1){4}}
    \put( -5,-6.5){\LINE(1,0){10}}
    \put( -5,-6.5){\LINE(0,-1){6}}
    \put( -5,-9.5){\GUARD{\INGUARD{y}}}
    \put(  5,-6.5){\LINE(0,-1){6}}
    \put(  5,-9.5){\GUARD{\INGUARD{not y}}}
    \put( 2,0){\LINE(1,0){2}}
    \STEP[$s_3$]{\INSTEP{x\,:=\,not x}}
    }
  % splitten im rechten zweig
  \put( 7, 25){\put(  0,-2.5){\LINE(0,-1){6}}
    \put(  0,-5.5){\GUARD}
    \STEP[$s_5$]{}
    }
  \put( 17, 25){\put(  0,-2.5){\LINE(0,-1){6}}
    \put(  0,-5.5){\GUARD}
    \STEP[$s_6$]{}
    }
  % zusammenfuehrung rechts nach splitten
  \put( 7, 19){\put(  0,-2.5){\LINE(1,0){10}}
    \put(  5,-2.5){\LINE(0,-1){4}}
    }
  \put(  12, 10){
    \STEP[$s_7$]{}
    \put(  0,-2.5){\LINE(0,-1){4}}
    }

  % unterer step im linken zweig + parallel-konstrukt schliessen
  \put(-12, 10){\put(  0,-2.5){\LINE(0,-1){4}}
    \put(  0,-5.5){\LINE(1,0){24}}
    \put(  0,-6.5){\LINE(1,0){24}}
    \put( 12,-6.5){\LINE(0,-1){6}}
    \put( 12,-9.5){\GUARD}
    \STEP[$s_4$]{}
    }
  % unterster step + nach oben
  \put(  0, -5){\put(  0,-2.5){\LINE(0,-1){6}}
    \put(  0,-5.5){\GUARD}
    \put(  0,-8.5){\LINE(-1,0){20}}
    \put(-20,-8.5){\LINE(0,1){68.5}}
    \put(-20,60  ){\VECTOR(1,0){17.5}}
    \STEP[$s_8$]{}
    }
\end{picture}


\clearpage


\Pfoilhead{SFC's (3)}

\begin{itemize}
\item Steps = Knoten, assoziiert dazu sind Aktionen
  (z.B. \texttt{x\,:=\,false})
\item Transitionen mit Guards zwischen Steps
\end{itemize}

\bigskip

\begin{center}
  \textbf{\large Arbeitszyklus}
\end{center}

\begin{itemize}
\item Inputs lesen von der Umgebung
\item Aktionen der aktiven Steps ausf�hren
\item Guards auswerten und Transitionen nehmen (wenn m�glich)
\item Outputs schreiben
\end{itemize}


\clearpage


\Pfoilhead{Slime}

Entwicklungsumgebung f�r SFC's, bestehend aus den Paketen
\begin{itemize}
\item GUI (graphische Benutzerschnittstelle)
\item Editor
\item Platzierung
\item Parser
\item Checks
\item Simulator
\item �bersetzung nach SMV
\item Model-Checker
\item Codegenerierung
\item Hilfsprogramme (z.B. pretty-printer)
\end{itemize}
 


\end{document}


%%% Local Variables: 
%%% mode: latex
%%% TeX-master: t
%%% End: 
